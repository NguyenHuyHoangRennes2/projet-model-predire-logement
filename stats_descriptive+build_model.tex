% Options for packages loaded elsewhere
\PassOptionsToPackage{unicode}{hyperref}
\PassOptionsToPackage{hyphens}{url}
%
\documentclass[
]{article}
\usepackage{amsmath,amssymb}
\usepackage{iftex}
\ifPDFTeX
  \usepackage[T1]{fontenc}
  \usepackage[utf8]{inputenc}
  \usepackage{textcomp} % provide euro and other symbols
\else % if luatex or xetex
  \usepackage{unicode-math} % this also loads fontspec
  \defaultfontfeatures{Scale=MatchLowercase}
  \defaultfontfeatures[\rmfamily]{Ligatures=TeX,Scale=1}
\fi
\usepackage{lmodern}
\ifPDFTeX\else
  % xetex/luatex font selection
\fi
% Use upquote if available, for straight quotes in verbatim environments
\IfFileExists{upquote.sty}{\usepackage{upquote}}{}
\IfFileExists{microtype.sty}{% use microtype if available
  \usepackage[]{microtype}
  \UseMicrotypeSet[protrusion]{basicmath} % disable protrusion for tt fonts
}{}
\makeatletter
\@ifundefined{KOMAClassName}{% if non-KOMA class
  \IfFileExists{parskip.sty}{%
    \usepackage{parskip}
  }{% else
    \setlength{\parindent}{0pt}
    \setlength{\parskip}{6pt plus 2pt minus 1pt}}
}{% if KOMA class
  \KOMAoptions{parskip=half}}
\makeatother
\usepackage{xcolor}
\usepackage[margin=1in]{geometry}
\usepackage{color}
\usepackage{fancyvrb}
\newcommand{\VerbBar}{|}
\newcommand{\VERB}{\Verb[commandchars=\\\{\}]}
\DefineVerbatimEnvironment{Highlighting}{Verbatim}{commandchars=\\\{\}}
% Add ',fontsize=\small' for more characters per line
\usepackage{framed}
\definecolor{shadecolor}{RGB}{248,248,248}
\newenvironment{Shaded}{\begin{snugshade}}{\end{snugshade}}
\newcommand{\AlertTok}[1]{\textcolor[rgb]{0.94,0.16,0.16}{#1}}
\newcommand{\AnnotationTok}[1]{\textcolor[rgb]{0.56,0.35,0.01}{\textbf{\textit{#1}}}}
\newcommand{\AttributeTok}[1]{\textcolor[rgb]{0.13,0.29,0.53}{#1}}
\newcommand{\BaseNTok}[1]{\textcolor[rgb]{0.00,0.00,0.81}{#1}}
\newcommand{\BuiltInTok}[1]{#1}
\newcommand{\CharTok}[1]{\textcolor[rgb]{0.31,0.60,0.02}{#1}}
\newcommand{\CommentTok}[1]{\textcolor[rgb]{0.56,0.35,0.01}{\textit{#1}}}
\newcommand{\CommentVarTok}[1]{\textcolor[rgb]{0.56,0.35,0.01}{\textbf{\textit{#1}}}}
\newcommand{\ConstantTok}[1]{\textcolor[rgb]{0.56,0.35,0.01}{#1}}
\newcommand{\ControlFlowTok}[1]{\textcolor[rgb]{0.13,0.29,0.53}{\textbf{#1}}}
\newcommand{\DataTypeTok}[1]{\textcolor[rgb]{0.13,0.29,0.53}{#1}}
\newcommand{\DecValTok}[1]{\textcolor[rgb]{0.00,0.00,0.81}{#1}}
\newcommand{\DocumentationTok}[1]{\textcolor[rgb]{0.56,0.35,0.01}{\textbf{\textit{#1}}}}
\newcommand{\ErrorTok}[1]{\textcolor[rgb]{0.64,0.00,0.00}{\textbf{#1}}}
\newcommand{\ExtensionTok}[1]{#1}
\newcommand{\FloatTok}[1]{\textcolor[rgb]{0.00,0.00,0.81}{#1}}
\newcommand{\FunctionTok}[1]{\textcolor[rgb]{0.13,0.29,0.53}{\textbf{#1}}}
\newcommand{\ImportTok}[1]{#1}
\newcommand{\InformationTok}[1]{\textcolor[rgb]{0.56,0.35,0.01}{\textbf{\textit{#1}}}}
\newcommand{\KeywordTok}[1]{\textcolor[rgb]{0.13,0.29,0.53}{\textbf{#1}}}
\newcommand{\NormalTok}[1]{#1}
\newcommand{\OperatorTok}[1]{\textcolor[rgb]{0.81,0.36,0.00}{\textbf{#1}}}
\newcommand{\OtherTok}[1]{\textcolor[rgb]{0.56,0.35,0.01}{#1}}
\newcommand{\PreprocessorTok}[1]{\textcolor[rgb]{0.56,0.35,0.01}{\textit{#1}}}
\newcommand{\RegionMarkerTok}[1]{#1}
\newcommand{\SpecialCharTok}[1]{\textcolor[rgb]{0.81,0.36,0.00}{\textbf{#1}}}
\newcommand{\SpecialStringTok}[1]{\textcolor[rgb]{0.31,0.60,0.02}{#1}}
\newcommand{\StringTok}[1]{\textcolor[rgb]{0.31,0.60,0.02}{#1}}
\newcommand{\VariableTok}[1]{\textcolor[rgb]{0.00,0.00,0.00}{#1}}
\newcommand{\VerbatimStringTok}[1]{\textcolor[rgb]{0.31,0.60,0.02}{#1}}
\newcommand{\WarningTok}[1]{\textcolor[rgb]{0.56,0.35,0.01}{\textbf{\textit{#1}}}}
\usepackage{longtable,booktabs,array}
\usepackage{calc} % for calculating minipage widths
% Correct order of tables after \paragraph or \subparagraph
\usepackage{etoolbox}
\makeatletter
\patchcmd\longtable{\par}{\if@noskipsec\mbox{}\fi\par}{}{}
\makeatother
% Allow footnotes in longtable head/foot
\IfFileExists{footnotehyper.sty}{\usepackage{footnotehyper}}{\usepackage{footnote}}
\makesavenoteenv{longtable}
\usepackage{graphicx}
\makeatletter
\def\maxwidth{\ifdim\Gin@nat@width>\linewidth\linewidth\else\Gin@nat@width\fi}
\def\maxheight{\ifdim\Gin@nat@height>\textheight\textheight\else\Gin@nat@height\fi}
\makeatother
% Scale images if necessary, so that they will not overflow the page
% margins by default, and it is still possible to overwrite the defaults
% using explicit options in \includegraphics[width, height, ...]{}
\setkeys{Gin}{width=\maxwidth,height=\maxheight,keepaspectratio}
% Set default figure placement to htbp
\makeatletter
\def\fps@figure{htbp}
\makeatother
\setlength{\emergencystretch}{3em} % prevent overfull lines
\providecommand{\tightlist}{%
  \setlength{\itemsep}{0pt}\setlength{\parskip}{0pt}}
\setcounter{secnumdepth}{-\maxdimen} % remove section numbering
\usepackage{booktabs}
\usepackage{longtable}
\usepackage{array}
\usepackage{multirow}
\usepackage{wrapfig}
\usepackage{float}
\usepackage{colortbl}
\usepackage{pdflscape}
\usepackage{tabu}
\usepackage{threeparttable}
\usepackage{threeparttablex}
\usepackage[normalem]{ulem}
\usepackage{makecell}
\usepackage{xcolor}
\ifLuaTeX
  \usepackage{selnolig}  % disable illegal ligatures
\fi
\usepackage{bookmark}
\IfFileExists{xurl.sty}{\usepackage{xurl}}{} % add URL line breaks if available
\urlstyle{same}
\hypersetup{
  pdftitle={Analyse des déterminants du loyer à Rennes},
  pdfauthor={Nguyen Huy Hoang et EL GOURAINI Tao Loup},
  hidelinks,
  pdfcreator={LaTeX via pandoc}}

\title{Analyse des déterminants du loyer à Rennes}
\author{Nguyen Huy Hoang et EL GOURAINI Tao Loup}
\date{2025-12-06}

\begin{document}
\maketitle

\textbf{NOM - Prénom} \textbf{EL GOURAINI Tao Loup}
\href{https://eco.univ-rennes.fr/master-mathematiques-appliquees-statistique}{\textbf{Master
1 MAS - Université de Rennes}}

\textbf{Nguyen Huy Hoang}
\href{https://formations.univ-rennes2.fr/fr/formations/master-37/master-mention-mathematiques-appliquees-statistique-parcours-sciences-des-donnees-intelligence-artificielle-JFTJBMKM.html}{\textbf{Master
1 MAS - Université Rennes 2}}

\section{1. Introduction}\label{introduction}

L'accès au logement constitue aujourd'hui un enjeu socio-économique
majeur en France. Dans un contexte de pression croissante sur le marché
locatif, notamment dans les villes à forte attractivité universitaire
comme Rennes, les ménages sont confrontés à une hausse soutenue des
loyers. Comprendre les facteurs qui influencent le prix des logements
devient alors essentiel pour analyser les disparités territoriales et
éclairer les décisions des acteurs publics.

Dans ce projet, nous cherchons à analyser empiriquement les
\textbf{déterminants du loyer tout compris} des logements situés à
Rennes, en utilisant des données collectées automatiquement par web
scraping sur le site \emph{Ouest-France Immo}. L'étude se concentre sur
les logements qui sont répartis dans quatre grands secteurs de Rennes :
\textbf{Centre}, \textbf{Ouest}, \textbf{Nord-Est} et
\textbf{Nord-Ouest}.

\subsection{1.1 Contexte et
problématique}\label{contexte-et-probluxe9matique}

Le marché rennais est particulièrement dynamique en raison de sa
croissance démographique, de son attractivité étudiante et d'une offre
locative limitée. Dès lors, une question centrale se pose :

\begin{quote}
\textbf{Quels facteurs structurels expliquent les variations de loyer à
Rennes ?}
\end{quote}

Cela conduit à examiner l'effet de la surface, du nombre de pièces, des
charges, de la performance énergétique et de la localisation
géographique.

\subsection{1.2 Motivation et intérêt du
sujet}\label{motivation-et-intuxe9ruxeat-du-sujet}

Ce travail présente plusieurs intérêts : - illustrer une application
réelle de la \textbf{régression linéaire} à partir de données collectées
soi-même ; - analyser les \textbf{disparités intra-urbaines} du marché
locatif rennais ; - évaluer l'influence de caractéristiques
structurelles ou environnementales du logement ; - proposer un exemple
complet de démarche allant de la collecte de données au modèle final.

\subsection{1.3 Questions de recherche et
hypothèses}\label{questions-de-recherche-et-hypothuxe8ses}

Les questions principales sont :

\begin{enumerate}
\def\labelenumi{\arabic{enumi}.}
\tightlist
\item
  La surface influence-t-elle fortement le montant du loyer ?
\item
  Les charges intégrées modifient-elles significativement le niveau du
  loyer tout compris ?
\item
  La performance énergétique joue-t-elle un rôle dans la détermination
  du prix ?
\item
  Le quartier exerce-t-il un effet propre sur les niveaux de loyer ?
\end{enumerate}

Les hypothèses qui en découlent sont :

\begin{itemize}
\item
  \textbf{H1 :} une surface plus élevée entraîne un loyer plus élevé
\item
  \textbf{H2 :} des charges plus importantes augmentent le loyer tout
  compris
\item
  \textbf{H3 :} une meilleure performance énergétique est associée à un
  loyer supérieur
\item
  \textbf{H4 :} les logements du Centre sont plus chers que ceux des
  autres quartiers.
\end{itemize}

\subsection{1.4 Objectifs du mémoire}\label{objectifs-du-muxe9moire}

Les objectifs du projet sont : - modéliser et expliquer le loyer mensuel
tout compris ; - identifier les déterminants les plus significatifs ; -
analyser les différences selon les quartiers rennais ; - proposer une
interprétation économétrique rigoureuse.

\subsection{1.5 Plan du document}\label{plan-du-document}

Ce mémoire est structuré comme suit :

\begin{itemize}
\tightlist
\item
  \textbf{Section 1 : Introduction}
\item
  \textbf{Section 2 : Cadre méthodologique}
\item
  \textbf{Section 3 : Analyse descriptive des données}
\item
  \textbf{Section 4 : Estimation et interprétation du modèle}
\item
  \textbf{Section 5 : Conclusion}
\end{itemize}

\begin{center}\rule{0.5\linewidth}{0.5pt}\end{center}

\section{2. Cadre méthodologique}\label{cadre-muxe9thodologique}

Cette section présente le modèle économétrique utilisé et les choix
méthodologiques qui guident l'estimation.

\subsection{2.1 Choix méthodologiques : régression linéaire et
MCO}\label{choix-muxe9thodologiques-ruxe9gression-linuxe9aire-et-mco}

Le modèle retenu est une \textbf{régression linéaire multiple}, estimée
à l'aide de la méthode des \textbf{Moindres Carrés Ordinaires (MCO)}.
Cette approche permet d'étudier comment une variable quantitative , ici
le loyer tout compris dépend de plusieurs caractéristiques propres au
logement.

La méthode MCO consiste à minimiser la somme des carrés des écarts entre
les valeurs observées et celles prédites par le modèle. Cette méthode
est privilégiée pour sa simplicité, son interprétabilité et les
propriétés de ses estimateurs.

\subsection{2.2 Formulation du modèle}\label{formulation-du-moduxe8le}

La variable expliquée est le \textbf{loyer mensuel tout compris} (en
euros).

Les variables explicatives sont :

\begin{itemize}
\tightlist
\item
  la \textbf{surface} du logement (m²),
\item
  les \textbf{charges comprises} (euros),
\item
  le \textbf{nombre de pièces},
\item
  la \textbf{classe énergétique} (A, B, C, \ldots),
\item
  la \textbf{consommation énergétique annuelle} (kWh/an),
\item
  les \textbf{émissions de CO₂} (kg/an),
\item
  le \textbf{quartier} (Centre, Ouest, Nord-Est, Nord-Ouest).
\end{itemize}

Le modèle général s'écrit :

\[
\text{Loyer}_i = \beta_0 +
\beta_1 \text{Surface}_i +
\beta_2 \text{Charges}_i +
\beta_3 \text{Pièces}_i +
\beta_4 \text{ClasseÉnergie}_i +
\beta_5 \text{Consommation}_i +
\beta_6 \text{ÉmissionsCO2}_i +
\sum_q \gamma_q \text{Quartier}_{qi} +
\varepsilon_i
\]

où : - \(\beta_0\) est la constante

\begin{itemize}
\item
  \(\beta_j\) mesure l'effet marginal de chaque caractéristique
\item
  \(\gamma_q\) capture l'effet des quartiers par rapport à un quartier
  de référence
\item
  \(\varepsilon_i\) est le terme d'erreur
\end{itemize}

\subsection{2.3 Justification des choix
techniques}\label{justification-des-choix-techniques}

Plusieurs éléments motivent ce cadre :

\begin{itemize}
\tightlist
\item
  La régression linéaire est un outil adapté pour analyser les facteurs
  influençant une variable économique.
\item
  La méthode MCO fournit des estimateurs robustes et facilement
  interprétables.
\item
  L'utilisation de \textbf{variables indicatrices} pour les quartiers
  permet de mettre en évidence les différences géographiques de prix.
\item
  L'intégration des variables énergétiques répond aux enjeux actuels de
  performance environnementale.
\item
  Le modèle permet de mesurer l'effet propre de chaque variable tout en
  contrôlant les autres.
\end{itemize}

\begin{center}\rule{0.5\linewidth}{0.5pt}\end{center}

\section{3. Analyse des données}\label{analyse-des-donnuxe9es}

\subsection{3.1 Cadre des données}\label{cadre-des-donnuxe9es}

Les données proviennent du site \textbf{Ouest-France Immo}, collectées à
l'aide d'un script de web scraping. Elles décrivent des logements mis en
location dans la \textbf{ville de Rennes}, .

L'étude porte sur \textbf{quatre secteurs géographiques} :

\begin{itemize}
\item
  Quartiers Centre
\item
  Quartiers Ouest
\item
  Quartiers Nord-Est
\item
  Quartiers Nord-Ouest
\end{itemize}

Chaque ligne du jeu de données correspond à \textbf{une annonce
unique}.\\
Les variables collectées décrivent :

\begin{itemize}
\item
  Les caractéristiques du logement (surface, nombre de pièces, charges)
\item
  Sa performance énergétique (classe, consommation, émissions)
\item
  Sa localisation.
\end{itemize}

\subsubsection{a) Importation des
données}\label{a-importation-des-donnuxe9es}

\begin{verbatim}
##   Loyer_TCC Surface Charges Pièces Pièce label_eco kWh kgCO2
## 1       278      19      NA     NA     1         D 247    49
## 2       343      17      15     NA     1         E 322     9
## 3       345       9      80     NA     1         C 230     7
## 4       350      17      10     NA     1         D 248    40
## 5       359      15      20     NA     1         F 367    21
##              Quartiers
## 1   Quartiers Nord-Est
## 2   Quartiers Nord-Est
## 3 Quartiers Nord-Ouest
## 4 Quartiers Nord-Ouest
## 5     Quartiers Centre
\end{verbatim}

\subsubsection{b) Description des
variables}\label{b-description-des-variables}

\begin{longtable}[]{@{}
  >{\raggedright\arraybackslash}p{(\columnwidth - 6\tabcolsep) * \real{0.1700}}
  >{\raggedright\arraybackslash}p{(\columnwidth - 6\tabcolsep) * \real{0.3400}}
  >{\raggedright\arraybackslash}p{(\columnwidth - 6\tabcolsep) * \real{0.1600}}
  >{\raggedright\arraybackslash}p{(\columnwidth - 6\tabcolsep) * \real{0.3300}}@{}}
\caption{Tableau 1 : Description des variables}\tabularnewline
\toprule\noalign{}
\begin{minipage}[b]{\linewidth}\raggedright
Code.de.la.série
\end{minipage} & \begin{minipage}[b]{\linewidth}\raggedright
Définition
\end{minipage} & \begin{minipage}[b]{\linewidth}\raggedright
Unité
\end{minipage} & \begin{minipage}[b]{\linewidth}\raggedright
Source
\end{minipage} \\
\midrule\noalign{}
\endfirsthead
\toprule\noalign{}
\begin{minipage}[b]{\linewidth}\raggedright
Code.de.la.série
\end{minipage} & \begin{minipage}[b]{\linewidth}\raggedright
Définition
\end{minipage} & \begin{minipage}[b]{\linewidth}\raggedright
Unité
\end{minipage} & \begin{minipage}[b]{\linewidth}\raggedright
Source
\end{minipage} \\
\midrule\noalign{}
\endhead
\bottomrule\noalign{}
\endlastfoot
Loyer\_TCC & Loyer mensuel tout compris & Euros & Ouest-France Immo (web
scraping) \\
Surface & Surface habitable du logement & m² & Ouest-France Immo (web
scraping) \\
Charges & Montant des charges mensuelles & Euros & Ouest-France Immo
(web scraping) \\
Pieces & Nombre de pièces du logement & Nombre & Ouest-France Immo (web
scraping) \\
label\_eco & Classe énergétique du logement & Catégorie (A--G) &
Ouest-France Immo (web scraping) \\
kWh & Consommation énergétique annuelle & kWh/an & Ouest-France Immo
(web scraping) \\
kgCO2 & Émissions annuelles de CO2 & kg/an & Ouest-France Immo (web
scraping) \\
Quartiers & Secteur géographique du logement & Nom du quartier &
Ouest-France Immo (web scraping) \\
\end{longtable}

\subsection{3.2 Prétraitement et nettoyage des
données}\label{pruxe9traitement-et-nettoyage-des-donnuxe9es}

Avant de procéder aux analyses descriptives et économétriques, il a été
nécessaire de réaliser un travail approfondi de préparation des données
issues du web scraping. Comme souvent avec des données collectées
directement sur un site d'annonces, plusieurs problèmes se présentent :
valeurs manquantes, incohérences, doublons, hétérogénéité dans les
formats ou dans les catégories. Cette section présente de manière
structurée l'ensemble des opérations réalisées pour obtenir un jeu de
données exploitable.

\subsubsection{a)Sélection et renommage des
variables}\label{asuxe9lection-et-renommage-des-variables}

Dans un premier temps, seules les variables pertinentes pour l'étude ont
été conservées.\\
Les colonnes correspondant ``Loyer'', ``Surface.habitable'',
``Dont.charges'',``Pièces'' , ``Pièce'', ``label\_eco'',
``kWh.m\ldots an'',``kgCO2.m\ldots an'', ``Quartiers'' ont été
extraites.\\
Les noms ont également été standardisés pour faciliter leur manipulation
(par exemple : renommer \emph{Surface habitable} en \emph{Surface},
\emph{Dont charges} en \emph{Charges}, etc.).

\subsubsection{b) Corriger les valeurs manquantes et modéliser les
niveaux}\label{b-corriger-les-valeurs-manquantes-et-moduxe9liser-les-niveaux}

\paragraph{b.1 Modéliser les
niveaux}\label{b.1-moduxe9liser-les-niveaux}

\textbf{Découper le Loyer TCC en 4 facteurs }

\begin{Shaded}
\begin{Highlighting}[]
\NormalTok{seuil }\OtherTok{\textless{}{-}}  \FunctionTok{quantile}\NormalTok{(annonces}\SpecialCharTok{$}\NormalTok{Loyer\_TCC ,}\AttributeTok{probs =} \FunctionTok{seq}\NormalTok{(}\DecValTok{0}\NormalTok{, }\DecValTok{1}\NormalTok{, }\FloatTok{0.25}\NormalTok{))}
\NormalTok{annonces}\SpecialCharTok{$}\NormalTok{Loyer\_Groupe }\OtherTok{\textless{}{-}} \FunctionTok{cut}\NormalTok{(annonces}\SpecialCharTok{$}\NormalTok{Loyer\_TCC , }\AttributeTok{breaks =}\NormalTok{ seuil , }\AttributeTok{include.lowest =} \ConstantTok{TRUE}\NormalTok{)}
\FunctionTok{levels}\NormalTok{(annonces}\SpecialCharTok{$}\NormalTok{Loyer\_Groupe)}
\end{Highlighting}
\end{Shaded}

\begin{verbatim}
## [1] "[278,535]"        "(535,700]"        "(700,1e+03]"      "(1e+03,2.25e+03]"
\end{verbatim}

\textbf{Exclusion des annonces non classées énergétiquement} Les
logements portant la mention \emph{NC} (Non Classé) ont été retirés.

\textbf{Regroupement de certaines catégories énergétiques} : les labels
\emph{G}, \emph{A} , \emph{B} et \emph{C} ont été regroupés dans une
catégorie commune (\emph{C+}) et les labels \emph{E} et \emph{F} ont été
regroupés dans une catégorie commune (\emph{E-}) pour éviter des classes
trop peu représentées.

\begin{verbatim}
## [1] "C+" "D"  "E-"
\end{verbatim}

\begin{verbatim}
##  [1] 1 1 1 1 1 1 1 1 1 1 1 1 1 1 1 1 1 1 1 1 1 1 1 1 1 1 1 1 1 1 1 1 1 1 1 1 1 1
## [39] 1 1 1 1 1 1 1 1 1 1 1 1
\end{verbatim}

\textbf{Apres traitement}

\begin{verbatim}
## 
##  1  2  3 
## 50 73 58
\end{verbatim}

\paragraph{b.2 One Hot Encoding la variables
Quartiers}\label{b.2-one-hot-encoding-la-variables-quartiers}

\paragraph{b.3 Correction des valeurs
manquante}\label{b.3-correction-des-valeurs-manquante}

\begin{itemize}
\item
  \textbf{Correction des pièces manquantes} : Le jeu de données contient
  deux variables liées au nombre de pièces du logement :
\item
  \textbf{Pièces} : prend une valeur manquante (\texttt{NA}) ou bien un
  nombre de pièces \textbf{strictement supérieur à 1} ;
\item
  \textbf{Piece} : variable binaire prenant la valeur \texttt{1} ou
  \texttt{NA}, et renseignée uniquement lorsque \texttt{Pièces} est
  manquante. Finalement, nous avons remplacé toutes les valeurs
  manquantes par 1 (1 pièce).
\item
  \textbf{Imputation énergétique} : les valeurs manquantes concernant la
  consommation en kWh et les émissions de CO₂ ont été remplacées par la
  moyenne calculée au sein de chaque classe énergétique.\\
  Cela permet une imputation cohérente avec le profil énergétique du
  logement.
\end{itemize}

Ces opérations permettent d'éviter la perte excessive d'observations
tout en conservant une cohérence structurelle entre les variables.

\textbf{Remplacer des valeurs manquantes dans les charges}

Certaines annonces ne précisaient pas le montant des charges.\\
Plutôt qu'une suppression pure et simple, les charges manquantes ont été
estimées en utilisant une méthode par \textbf{groupes de loyers} :

\begin{enumerate}
\def\labelenumi{\arabic{enumi}.}
\tightlist
\item
  Le loyer total a été divisé en cinq intervalles (du plus faible au
  plus élevé).
\item
  Chaque logement a été associé à un groupe de loyer.
\item
  Les charges manquantes ont été remplacées par la moyenne observée dans
  leur groupe de loyers respectif.
\end{enumerate}

Cette méthode permet d'imputer les charges selon un critère économique
pertinent : les logements ayant

\subsubsection{c) Vérification des effectifs et cohérence
interne}\label{c-vuxe9rification-des-effectifs-et-cohuxe9rence-interne}

Les effectifs par label énergétique ont été inspectés pour vérifier la
répartition des logements après nettoyage.\\
Cette étape permet de s'assurer que les regroupements effectués ne
créent pas de déséquilibres majeurs ou de classes trop petites.

Le travail consiste également à vérifier la cohérence logique des
données, comme :

\begin{itemize}
\tightlist
\item
  absence de valeurs énergétiques impossibles,
\item
  absence de surfaces irréalistes,
\item
  cohérence entre loyer et charges.
\end{itemize}

Les observations incohérentes ont été retirées ou corrigées selon les
cas.

\textbf{Vérification l'absence de valeur manquante }

\begin{Shaded}
\begin{Highlighting}[]
\FunctionTok{anyNA}\NormalTok{(annonces)}
\end{Highlighting}
\end{Shaded}

\begin{verbatim}
## [1] FALSE
\end{verbatim}

\subsubsection{d) Constitution du jeu de données
final}\label{d-constitution-du-jeu-de-donnuxe9es-final}

Après l'ensemble de ces traitements, la base obtenue présente les
caractéristiques suivantes :

\begin{itemize}
\tightlist
\item
  absence de doublons,
\item
  absence de valeurs cruciales manquantes,
\item
  cohérence des formats (numérique, catégoriel, etc.),
\item
  regroupement homogène des catégories énergétiques,
\item
  estimation raisonnable des charges manquantes,
\item
  structure compatible avec les outils de modélisation et d'analyse
  statistique.
\end{itemize}

Cette base nettoyée constitue le fondement de l'analyse descriptive
(Section 3.3) et du modèle économétrique présenté dans la Section 4.

\subsection{3.3 Analyse descriptive univariée des
données}\label{analyse-descriptive-univariuxe9e-des-donnuxe9es}

Cette section présente une description statistique des variables
utilisées dans le modèle, après les étapes de nettoyage décrites
précédemment.\\
L'objectif est de caractériser la distribution de chacune des variables,
d'identifier les tendances générales et de repérer d'éventuelles valeurs
atypiques.\\
L'analyse s'appuie sur des tableaux de statistiques descriptives ainsi
que des représentations graphiques adaptées (histogrammes, boxplots,
diagrammes en barres).

\subsubsection{a) Résumé statistique}\label{a-ruxe9sumuxe9-statistique}

Les principales variables quantitatives du jeu de données sont :

\begin{itemize}
\tightlist
\item
  le \textbf{loyer total} (euros),
\item
  la \textbf{surface habitable} (m²),
\item
  les \textbf{charges mensuelles} (euros),
\item
  le \textbf{nombre de pièces},
\item
  la \textbf{consommation énergétique} (kWh/an),
\item
  les \textbf{émissions de CO₂} (kg/an).
\end{itemize}

Le tableau suivant présente les statistiques descriptives classiques :
moyenne, médiane, minimum, maximum et écart-type.

\begin{longtable}[]{@{}
  >{\raggedright\arraybackslash}p{(\columnwidth - 16\tabcolsep) * \real{0.1471}}
  >{\raggedright\arraybackslash}p{(\columnwidth - 16\tabcolsep) * \real{0.1471}}
  >{\raggedleft\arraybackslash}p{(\columnwidth - 16\tabcolsep) * \real{0.0588}}
  >{\raggedleft\arraybackslash}p{(\columnwidth - 16\tabcolsep) * \real{0.0588}}
  >{\raggedleft\arraybackslash}p{(\columnwidth - 16\tabcolsep) * \real{0.1029}}
  >{\raggedleft\arraybackslash}p{(\columnwidth - 16\tabcolsep) * \real{0.1618}}
  >{\raggedleft\arraybackslash}p{(\columnwidth - 16\tabcolsep) * \real{0.0735}}
  >{\raggedleft\arraybackslash}p{(\columnwidth - 16\tabcolsep) * \real{0.0735}}
  >{\raggedleft\arraybackslash}p{(\columnwidth - 16\tabcolsep) * \real{0.1765}}@{}}
\caption{Tableau 2 : Statistiques descriptives des variables
quantitatives}\tabularnewline
\toprule\noalign{}
\begin{minipage}[b]{\linewidth}\raggedright
\end{minipage} & \begin{minipage}[b]{\linewidth}\raggedright
Var
\end{minipage} & \begin{minipage}[b]{\linewidth}\raggedleft
Min
\end{minipage} & \begin{minipage}[b]{\linewidth}\raggedleft
Q1
\end{minipage} & \begin{minipage}[b]{\linewidth}\raggedleft
Median
\end{minipage} & \begin{minipage}[b]{\linewidth}\raggedleft
Mean
\end{minipage} & \begin{minipage}[b]{\linewidth}\raggedleft
Q3
\end{minipage} & \begin{minipage}[b]{\linewidth}\raggedleft
Max
\end{minipage} & \begin{minipage}[b]{\linewidth}\raggedleft
SD
\end{minipage} \\
\midrule\noalign{}
\endfirsthead
\toprule\noalign{}
\begin{minipage}[b]{\linewidth}\raggedright
\end{minipage} & \begin{minipage}[b]{\linewidth}\raggedright
Var
\end{minipage} & \begin{minipage}[b]{\linewidth}\raggedleft
Min
\end{minipage} & \begin{minipage}[b]{\linewidth}\raggedleft
Q1
\end{minipage} & \begin{minipage}[b]{\linewidth}\raggedleft
Median
\end{minipage} & \begin{minipage}[b]{\linewidth}\raggedleft
Mean
\end{minipage} & \begin{minipage}[b]{\linewidth}\raggedleft
Q3
\end{minipage} & \begin{minipage}[b]{\linewidth}\raggedleft
Max
\end{minipage} & \begin{minipage}[b]{\linewidth}\raggedleft
SD
\end{minipage} \\
\midrule\noalign{}
\endhead
\bottomrule\noalign{}
\endlastfoot
Loyer\_TCC & Loyer\_TCC & 278 & 550 & 710.00 & 818.348066 & 1030 & 2250
& 365.6034059 \\
Surface & Surface & 9 & 22 & 35.00 & 46.530387 & 68 & 135 &
30.0541255 \\
Charges & Charges & 10 & 45 & 68.43 & 75.611216 & 90 & 300 &
49.9080296 \\
Pièces & Pièces & 1 & 1 & 2.00 & 2.215470 & 3 & 5 & 1.3511244 \\
label\_eco & label\_eco & 1 & 1 & 2.00 & 2.044199 & 3 & 3 & 0.7733276 \\
kWh & kWh & 37 & 159 & 219.00 & 221.735878 & 282 & 739 & 97.0364703 \\
kgCO2 & kgCO2 & 1 & 9 & 19.00 & 25.051732 & 37 & 96 & 18.7276770 \\
\end{longtable}

\subsubsection{b) Statistiques descriptives des variables
catégorielles}\label{b-statistiques-descriptives-des-variables-catuxe9gorielles}

Les variables qualitatives retenues dans l'analyse sont :

\begin{itemize}
\item
  Le quartier du logement,
\item
  La classe énergétique.
\item
  Le tableau suivant présente la distribution des effectifs pour ces
  variables.
\item
  Le group de loyer apres la modélisation
\end{itemize}

\textbf{Interprétation } :

Les quatre tableaux obtenus après les étapes de nettoyage permettent
d'apprécier la structure du jeu de données final. De manière générale,
ils montrent que la base est désormais cohérente, équilibrée et prête
pour l'analyse statistique.

\begin{itemize}
\item
  \textbf{Répartition géographique}\\
  La distribution des logements entre les quatre secteurs de Rennes
  (Centre, Nord-Est, Nord-Ouest et Ouest) est relativement homogène.\\
  Aucun quartier n'est surreprésenté, ce qui garantit une analyse
  territoriale équilibrée et évite un biais lié à la localisation.
\item
  \textbf{Performance énergétique}\\
  Après la suppression des logements non classés (NC) et le regroupement
  des classes rares, les performances énergétiques se répartissent de
  manière équilibrée entre trois catégories principales (C+, D et E−).\\
  Cette restructuration permet d'obtenir des effectifs suffisants dans
  chaque classe pour une analyse économétrique fiable.
\item
  \textbf{Nombre de pièces}\\
  La majorité des annonces concernent des logements d'une pièce, ce qui
  est cohérent avec la forte présence étudiante à Rennes.\\
  Toutefois, les logements de deux à cinq pièces restent suffisamment
  représentés pour évaluer leur impact sur les variations de loyer.
\item
  \textbf{Groupes de loyers}\\
  La découpe des loyers en intervalles fournit des groupes de taille
  comparable, reflétant un marché locatif diversifié allant des petits
  logements économiques aux biens plus spacieux et onéreux.\\
  Cette structuration facilite les comparaisons et sert également de
  base à certaines imputations, notamment pour le montant des charges.
  \#\#\# c) Graphique des principales variables
\end{itemize}

L'histogramme permet de visualiser la distribution des variables
quantitatives. Les graphiques suivants représentent :

\includegraphics{stats_descriptive+build_model_files/figure-latex/unnamed-chunk-13-1.pdf}

\paragraph{La distribution des surfaces selon le groupe de
loyer}\label{la-distribution-des-surfaces-selon-le-groupe-de-loyer}

\textbf{Commentaire :}\\
On observe que la surface tend globalement à augmenter avec le groupe de
loyer.\\
Les logements appartenant au groupe de loyers le plus faible présentent
les surfaces les plus réduites, souvent inférieures à 25 m².\\
À l'inverse, les groupes de loyers les plus élevés se caractérisent par
des surfaces plus importantes, bien que la variabilité demeure élevée.

La présence de valeurs extrêmes dans les groupes intermédiaires suggère
que le loyer n'est pas seulement déterminé par la surface mais également
par d'autres facteurs tels que la localisation, la qualité du logement
ou la performance énergétique.

Cette analyse confirme toutefois une tendance générale : \textbf{les
logements plus grands sont associés à des niveaux de loyer plus élevés},
ce qui est cohérent avec les mécanismes du marché locatif.

\paragraph{La distribution des
charges}\label{la-distribution-des-charges}

\textbf{Commentaire : Analyse du nuage de points Charges -- Loyer}

\textbf{Relation globalement croissante}\\
On observe une tendance générale à la hausse : lorsque les charges
augmentent, le loyer total tend également à augmenter.\\
Cela est cohérent avec le fonctionnement du marché locatif :\\
- les logements plus grands ou mieux équipés ont généralement des
charges plus élevées (eau, chauffage collectif, ascenseur,
copropriété),\\
- et un loyer total plus important.

\textbf{Dispersion importante pour les faibles charges}\\
Pour des charges comprises entre 0 et 80 euros, les loyers apparaissent
très dispersés (de 350 € à plus de 1500 €).\\
Cela s'explique par le fait que deux logements ayant de faibles charges
peuvent différer fortement en surface, en localisation ou en standing.\\
Ainsi, les charges ne suffisent pas à elles seules pour expliquer le
montant du loyer.

\textbf{Présence de quelques valeurs extrêmes}\\
Quelques observations présentent des charges très élevées (200 à 300 €),
associées à des loyers supérieurs à 1500 €.\\
Ces cas correspondent probablement à :\\
- des appartements de grande taille,\\
- des résidences récentes ou disposant de services inclus,\\
- ou des logements haut de gamme.\\
Ces valeurs extrêmes restent peu nombreuses mais économiquement
cohérentes.

\textbf{Relation non linéaire et corrélation modérée}\\
La dispersion du nuage montre que les charges ne permettent pas à elles
seules de prédire précisément le loyer.\\
La relation est positive mais relativement faible.\\
Cela suggère que d'autres variables jouent un rôle déterminant : la
surface, le quartier, la classe énergétique, le nombre de pièces, etc.\\
Ceci justifie la mise en place d'un modèle de régression multiple
intégrant plusieurs caractéristiques du logement.

\paragraph{La distribution du prix de Loyer selon les
facteurs.}\label{la-distribution-du-prix-de-loyer-selon-les-facteurs.}

\includegraphics{stats_descriptive+build_model_files/figure-latex/unnamed-chunk-14-1.pdf}
\textbf{Commentaire : Analyse du loyer selon le nombre de pièces}

\textbf{Hausse nette du loyer avec le nombre de pièces}\\
Le boxplot montre une relation clairement croissante :\\
plus le logement comporte de pièces, plus le loyer tout compris
augmente.\\
Cette tendance est attendue, car les appartements plus spacieux
nécessitent un loyer plus élevé.

\textbf{Progression marquée entre chaque catégorie}\\
À chaque passage d'une catégorie de pièces à la suivante (1 → 2, 2 → 3,
etc.),\\
la médiane du loyer augmente significativement.\\
Cela confirme que le nombre de pièces constitue un déterminant majeur du
niveau de loyer.

\textbf{Dispersion croissante pour les grands logements}\\
Les logements de 3 pièces et plus présentent une plus grande variabilité
de prix :\\
- les biens peuvent varier fortement en surface,\\
- en standing,\\
- ou en localisation.\\
Cette diversité se traduit par des écarts-types plus élevés et quelques
valeurs extrêmes, notamment pour les 4 pièces.

\textbf{Présence d'outliers chez les petits et grands logements}\\
- Quelques studios (1 pièce) affichent des loyers anormalement élevés,
probablement liés à une localisation très centrale ou à des logements
meublés haut de gamme.\\
- À l'inverse, certains logements de 4 pièces présentent des loyers
exceptionnellement hauts (plus de 2000 €), cohérents avec des biens très
spacieux ou haut standing.

\textbf{Conclusion générale}\\
Le graphique confirme que \textbf{le nombre de pièces est un facteur
explicatif déterminant du loyer}.\\
Il influence à la fois le niveau central du loyer (médiane) et sa
dispersion.\\
Ce constat justifie l'intégration de cette variable dans la régression
linéaire comme variable explicative structurante.

\paragraph{La distribution du prix de Loyer selon
quartiers}\label{la-distribution-du-prix-de-loyer-selon-quartiers}

\textbf{Analyse du loyer selon les différents quartiers de Rennes}

\textbf{Quartiers Centre : les loyers les plus élevés}\\
Le quartier Centre présente les niveaux de loyer les plus hauts de
l'échantillon.\\
La médiane y dépasse nettement celle des autres secteurs, et la
dispersion est importante.\\
Cette variabilité reflète la diversité de l'offre dans l'hypercentre
(studios chers, logements rénovés, petites surfaces premium).

\textbf{Quartier Nord-Est : le secteur le plus abordable}\\
Le Nord-Est affiche les loyers les plus faibles, avec une médiane
sensiblement inférieure à celles des autres quartiers.\\
Ce secteur regroupe majoritairement des logements plus petits et plus
accessibles, ce qui explique son positionnement plus bas dans la
distribution.

\textbf{Quartier Nord-Ouest : un niveau intermédiaire mais dispersé}\\
Le Nord-Ouest présente une médiane de loyer proche de celle du Centre,
tout en restant légèrement inférieure.\\
La dispersion est cependant importante, signe d'une grande hétérogénéité
du parc locatif (logements étudiants, maisons familiales, rénovations
récentes).

\textbf{Quartier Ouest : des loyers modérés mais plus variés}\\
Le quartier Ouest se situe dans une position intermédiaire, avec une
médiane plus élevée que celle du Nord-Est mais inférieure à celle du
Centre et du Nord-Ouest.\\
Une dispersion notable indique la coexistence de logements modestes et
de biens plus spacieux ou récents.

\subsection{3.4 Analyse descriptive bivariée des
données}\label{analyse-descriptive-bivariuxe9e-des-donnuxe9es}

\subsubsection{Corrélation}\label{corruxe9lation}

\paragraph{1ère règle: Coefficient de
Corrélation}\label{uxe8re-ruxe8gle-coefficient-de-corruxe9lation}

\includegraphics{stats_descriptive+build_model_files/figure-latex/unnamed-chunk-15-1.pdf}
\includegraphics{stats_descriptive+build_model_files/figure-latex/unnamed-chunk-15-2.pdf}
\textbf{Commentaire : Corrélations entre le loyer et les variables
explicatives}

Le tableau met en évidence plusieurs relations importantes entre la
variable expliquée (\emph{Loyer\_TCC}) et les variables explicatives
quantitatives :

\begin{enumerate}
\def\labelenumi{\arabic{enumi}.}
\item
  \textbf{Surface (corr = 0.93)}\\
  C'est la corrélation la plus forte observée.\\
  Le loyer augmente très fortement avec la surface du logement.\\
  Cette relation quasi linéaire confirme que la surface est le principal
  déterminant du prix.
\item
  \textbf{Nombre de pièces (corr = 0.87)}\\
  Le nombre de pièces est également fortement corrélé au loyer.\\
  Cela s'explique naturellement par le fait qu'un logement comportant
  plus de pièces possède généralement une surface plus importante et un
  niveau de confort supérieur.
\item
  \textbf{Charges (corr = 0.61)}\\
  Une corrélation positive modérée :\\
  les logements ayant des charges plus élevées ont souvent des loyers
  plus élevés.\\
  Cependant, la relation est moins forte que pour la surface et les
  pièces.
\item
  \textbf{Consommation énergétique kWh (corr = --0.30)}\\
  La corrélation est faible et négative.\\
  Cela suggère que les logements très énergivores (kWh élevés) ont
  tendance à être légèrement moins chers, ce qui est cohérent avec
  l'idée qu'un mauvais bilan énergétique peut réduire la valeur
  locative.
\item
  \textbf{Émissions de CO₂ (corr = 0.21)}\\
  Corrélation faible et positive.\\
  Cela signifie qu'il n'existe pas de relation claire entre les
  émissions de CO₂ et le montant du loyer.\\
  Le CO₂ est probablement un facteur indirect qui reflète davantage la
  performance énergétique ou la taille du logement.
\end{enumerate}

Si la valeur absolue du coefficient de corrélation entre deux variables
excède 0.8, on peut soupçonner la colinéarité.

Selon cette règle, on peut soupçonner un problème de colinéarité entre
les variables vol et surface, ainsi que pers et surface.

\paragraph{Deuxième règle (règle de
Klein)}\label{deuxiuxe8me-ruxe8gle-ruxe8gle-de-klein}

\begin{verbatim}
## 
## Call:
## lm(formula = Loyer_TCC ~ Surface + Charges + Pièces + label_eco + 
##     kWh + kgCO2 + Quartiers_Centre + Quartiers_Nord_Est + Quartiers_Nord_Ouest + 
##     Quartiers_Ouest, data = annonces)
## 
## Residuals:
##     Min      1Q  Median      3Q     Max 
## -375.88  -73.58    1.08   73.79  696.48 
## 
## Coefficients: (1 not defined because of singularities)
##                       Estimate Std. Error t value Pr(>|t|)    
## (Intercept)           395.6343    38.9189  10.166  < 2e-16 ***
## Surface                 9.1852     1.0178   9.025 3.64e-16 ***
## Charges                 0.9140     0.2629   3.477 0.000643 ***
## Pièces                 23.6423    20.6380   1.146 0.253573    
## label_eco             -13.0154    19.8054  -0.657 0.511959    
## kWh                    -0.3097     0.1512  -2.048 0.042103 *  
## kgCO2                  -0.5603     0.5713  -0.981 0.328123    
## Quartiers_Centre       35.6670    27.2153   1.311 0.191766    
## Quartiers_Nord_Est    -15.2679    27.9475  -0.546 0.585568    
## Quartiers_Nord_Ouest -104.4316    29.5652  -3.532 0.000530 ***
## Quartiers_Ouest             NA         NA      NA       NA    
## ---
## Signif. codes:  0 '***' 0.001 '**' 0.01 '*' 0.05 '.' 0.1 ' ' 1
## 
## Residual standard error: 126.9 on 171 degrees of freedom
## Multiple R-squared:  0.8856, Adjusted R-squared:  0.8795 
## F-statistic:   147 on 9 and 171 DF,  p-value: < 2.2e-16
\end{verbatim}

Quartiers\_Ouest est une combinaison linéaire des autres (colinéarité
parfaite) → elle ne peut pas être estimée → d'où NA.

En supprimant la modalité, vous obtenez un modèle propre, cohérent et
sans singularités.

Si le carré du coefficient de corrélation est supérieur au R\^{}2, on
peut soupçonner de la colinéarité.

\begin{Shaded}
\begin{Highlighting}[]
\CommentTok{\#cor(vars\_quanti)\^{}2 \textgreater{} summary(reg)$r.squared}
\end{Highlighting}
\end{Shaded}

Cette règle met en avant un éventuel problème de colinéarité entre les
variables vol et surface.

\paragraph{Troisième règle : cohérence des
signes}\label{troisiuxe8me-ruxe8gle-cohuxe9rence-des-signes}

Le signe de la correlation brute et celui de la correlation expliquée
devraient être les memes.

\begin{Shaded}
\begin{Highlighting}[]
\CommentTok{\#cbind(coef = reg$coef, corr = cor(df)[,"kw"])[{-}1,]}
\end{Highlighting}
\end{Shaded}

Il y a manifestement des contre-sens concernant les relations entre le
nombre de personne et la consommation, ainsi que le nombre de salles de
bain et la consommation.

\paragraph{colinéarités : VIF}\label{colinuxe9arituxe9s-vif}

\begin{Shaded}
\begin{Highlighting}[]
\FunctionTok{library}\NormalTok{(car)}

\CommentTok{\#vif(reg)}

\CommentTok{\#commentaires}
\CommentTok{\#sqrt(vif(reg))}
\end{Highlighting}
\end{Shaded}

\subsubsection{b) Nuage de point vis à vis de
Y}\label{b-nuage-de-point-vis-uxe0-vis-de-y}

\begin{Shaded}
\begin{Highlighting}[]
\NormalTok{graphConso }\OtherTok{\textless{}{-}} \ControlFlowTok{function}\NormalTok{(uneVariable)\{}
  \FunctionTok{ggplot}\NormalTok{(}\AttributeTok{data =}\NormalTok{ annonces, }\FunctionTok{aes\_string}\NormalTok{(}\AttributeTok{x =}\NormalTok{ uneVariable, }\AttributeTok{y =} \StringTok{"Loyer\_TCC"}\NormalTok{)) }\SpecialCharTok{+} \FunctionTok{geom\_point}\NormalTok{() }\SpecialCharTok{+} \FunctionTok{geom\_smooth}\NormalTok{()}
\NormalTok{\}}

\NormalTok{p1 }\OtherTok{\textless{}{-}} \FunctionTok{graphConso}\NormalTok{(}\StringTok{"Surface"}\NormalTok{)}
\NormalTok{p2 }\OtherTok{\textless{}{-}} \FunctionTok{graphConso}\NormalTok{(}\StringTok{"Charges"}\NormalTok{)}
\NormalTok{p3 }\OtherTok{\textless{}{-}} \FunctionTok{graphConso}\NormalTok{(}\StringTok{"Pièces"}\NormalTok{)}
\NormalTok{p4 }\OtherTok{\textless{}{-}} \FunctionTok{graphConso}\NormalTok{(}\StringTok{"label\_eco"}\NormalTok{)}
\NormalTok{p5 }\OtherTok{\textless{}{-}} \FunctionTok{graphConso}\NormalTok{(}\StringTok{"kWh"}\NormalTok{)}
\NormalTok{p6 }\OtherTok{\textless{}{-}} \FunctionTok{graphConso}\NormalTok{(}\StringTok{"kgCO2"}\NormalTok{)}

\NormalTok{p7 }\OtherTok{\textless{}{-}}\FunctionTok{graphConso}\NormalTok{(}\StringTok{"Quartiers\_Centre"}\NormalTok{)}
\NormalTok{p8 }\OtherTok{\textless{}{-}}\FunctionTok{graphConso}\NormalTok{(}\StringTok{"Quartiers\_Nord\_Est"}\NormalTok{)}
\NormalTok{p9 }\OtherTok{\textless{}{-}}\FunctionTok{graphConso}\NormalTok{(}\StringTok{"Quartiers\_Nord\_Ouest"}\NormalTok{)}
\NormalTok{p10 }\OtherTok{\textless{}{-}}\FunctionTok{graphConso}\NormalTok{(}\StringTok{"Quartiers\_Ouest"}\NormalTok{)}

\FunctionTok{grid.arrange}\NormalTok{(p1, p2, p3, p4, p5, p6 , p7,p8,p9,p10, }\AttributeTok{ncol =} \DecValTok{3}\NormalTok{)}
\end{Highlighting}
\end{Shaded}

\includegraphics{stats_descriptive+build_model_files/figure-latex/unnamed-chunk-20-1.pdf}

\subsection{Analyse graphique de la relation entre le loyer et les
variables
explicatives}\label{analyse-graphique-de-la-relation-entre-le-loyer-et-les-variables-explicatives}

L'observation des nuages de points permet d'étudier la nature de la
relation entre le loyer toutes charges comprises (\textbf{Loyer\_TCC})
et les différentes variables explicatives disponibles dans la base de
données.\\
Les courbes de lissage (LOESS) permettent d'identifier si la relation
semble linéaire, croissante, décroissante ou non linéaire.

Les conclusions suivantes peuvent être formulées :

\subsubsection{\texorpdfstring{\textbf{Surface}}{Surface}}\label{surface}

La relation entre la surface et le loyer apparaît clairement
\textbf{croissante et linéaire}.\\
On peut donc modéliser cette relation par une forme linéaire simple : \[
Loyer\_{TCC} = \beta_0 + \beta_1 \cdot Surface
\]

\subsubsection{\texorpdfstring{\textbf{Charges}}{Charges}}\label{charges}

La relation entre les charges et le loyer présente une légère
\textbf{courbure}, suggérant une relation \textbf{quadratique} : \[
Loyer\_{TCC} = \beta_0 + \beta_1 \cdot Charges + \beta_2 \cdot Charges^2
\]

\subsubsection{\texorpdfstring{\textbf{Nombre de
pièces}}{Nombre de pièces}}\label{nombre-de-piuxe8ces}

La relation est \textbf{croissante et linéaire}, sans signe de
non-linéarité : \[
Loyer\_{TCC} = \beta_0 + \beta_1 \cdot Pièces
\]

\subsubsection{\texorpdfstring{\textbf{Classe énergétique
(label\_eco)}}{Classe énergétique (label\_eco)}}\label{classe-uxe9nerguxe9tique-label_eco}

La variable \emph{label\_eco} est qualitative : les points se
répartissent en bandes verticales.\\
Aucune tendance fonctionnelle continue ne peut être modélisée.\\
Elle doit être introduite dans le modèle sous forme de \textbf{variable
factorielle} : \[
Loyer\_{TCC} = \beta_0 + \sum_i \beta_i \cdot \mathbf{1}\{label\_eco = i\}
\]

\subsubsection{\texorpdfstring{\textbf{Consommation énergétique
(kWh)}}{Consommation énergétique (kWh)}}\label{consommation-uxe9nerguxe9tique-kwh}

La courbe montre une légère forme en « U ».\\
La relation peut donc être modélisée par une fonction
\textbf{quadratique} : \[
Loyer\_{TCC} = \beta_0 + \beta_1 \cdot kWh + \beta_2 \cdot kWh^2
\]

\subsubsection{\texorpdfstring{\textbf{Émissions de CO2
(kgCO2)}}{Émissions de CO2 (kgCO2)}}\label{uxe9missions-de-co2-kgco2}

Même comportement que pour kWh, avec une légère convexité : \[
Loyer\_{TCC} = \beta_0 + \beta_1 \cdot kgCO2 + \beta_2 \cdot kgCO2^2
\]

\subsubsection{\texorpdfstring{\textbf{Variables géographiques (Centre,
Nord\_Est, Nord\_Ouest,
Ouest)}}{Variables géographiques (Centre, Nord\_Est, Nord\_Ouest, Ouest)}}\label{variables-guxe9ographiques-centre-nord_est-nord_ouest-ouest}

Ces variables sont binaires (0/1).\\
Elles n'induisent pas de forme fonctionnelle mais représentent un
\textbf{effet de présence} du quartier correspondant : \[
Loyer\_{TCC} = \beta_0 + \beta_1 \cdot Quartier\_dummy
\]

\begin{center}\rule{0.5\linewidth}{0.5pt}\end{center}

\subsubsection{c) Corrélations entre les variables
explicatives}\label{c-corruxe9lations-entre-les-variables-explicatives}

L'analyse de la matrice de corrélation entre les variables explicatives
permet de détecter d'éventuels problèmes de multicolinéarité dans le
futur modèle de régression.

\begin{enumerate}
\def\labelenumi{\arabic{enumi}.}
\item
  \textbf{Surface et nombre de pièces (corr = 0.94)}\\
  Il s'agit de la corrélation la plus élevée entre les variables
  explicatives.\\
  Cela s'explique par le fait que les logements comportant davantage de
  pièces sont presque toujours plus grands.\\
  Cette corrélation très forte peut poser un risque de multicolinéarité
  dans la régression. Le modèle devra en tenir compte, par exemple en
  n'interprétant pas séparément ces deux variables ou en examinant les
  VIF.
\item
  \textbf{Surface et Charges (corr = 0.61)}\\
  Les logements plus grands ont en général des charges plus élevées
  (chauffage collectif, copropriété, entretien).\\
  La corrélation est modérée mais non problématique.
\item
  \textbf{Pièces et Charges (corr = 0.54)}\\
  Les appartements ayant plus de pièces entraînent généralement plus de
  charges.\\
  Cette corrélation reste raisonnable et ne conduit pas à une
  multicolinéarité forte.
\item
  \textbf{Variables énergétiques (kWh, CO₂)}

  \begin{itemize}
  \tightlist
  \item
    kWh présente de faibles corrélations avec les autres variables
    (entre −0.30 et −0.20).\\
  \item
    kgCO₂ montre également des corrélations faibles (entre 0.18 et
    0.25).
  \end{itemize}

  Ces valeurs indiquent que la performance énergétique évolue de manière
  relativement indépendante des autres caractéristiques du logement. Il
  n'existe donc pas de risque de multicolinéarité majeure provenant des
  variables énergétiques.
\item
  \textbf{Absence de corrélations problématiques en dehors de
  Surface--Pièces}\\
  Globalement, toutes les corrélations sont faibles ou modérées, à
  l'exception du couple Surface--Pièces, qui présente une colinéarité
  naturellement élevée mais interprétable.
\end{enumerate}

\textbf{Conclusion :}\\
La seule multicolinéarité forte concerne \emph{Surface} et
\emph{Pièces}.\\
Cela devra être vérifié lors de l'estimation du modèle (par les VIF),
mais les autres variables explicatives sont suffisamment indépendantes
pour être intégrées sans risque dans la régression linéaire.

\section{\texorpdfstring{\textbf{4. Analyse et discussion des
résultats}}{4. Analyse et discussion des résultats}}\label{analyse-et-discussion-des-ruxe9sultats}

\subsection{4.1 Modélisation
principale}\label{moduxe9lisation-principale}

Présentation et interprétation des résultats obtenus : utiliser un
tableau récapitulatif des modèles principaux testés

stepAIC() choisit le meilleur par AIC. En option on peut choisir la
direction:

\begin{enumerate}
\def\labelenumi{\roman{enumi})}
\tightlist
\item
  ``both'' (for stepwise regression, both forward and backward
  selection);
\item
  ``backward'' (for backward selection)
\item
  ``forward'' (for forward selection).
\end{enumerate}

En fonction de la procédure nous pouvons obtenir une sélection de
variables différentes

\begin{verbatim}
## 
## Call:
## lm(formula = Loyer_TCC ~ Surface + Charges + kWh + Quartiers_Centre + 
##     Quartiers_Nord_Ouest, data = annonces)
## 
## Residuals:
##     Min      1Q  Median      3Q     Max 
## -397.23  -73.93   -4.26   73.10  704.06 
## 
## Coefficients:
##                      Estimate Std. Error t value Pr(>|t|)    
## (Intercept)          377.8952    33.7860  11.185  < 2e-16 ***
## Surface               10.1402     0.4325  23.444  < 2e-16 ***
## Charges                0.8222     0.2571   3.199 0.001640 ** 
## kWh                   -0.3973     0.1053  -3.775 0.000219 ***
## Quartiers_Centre      44.5086    23.5151   1.893 0.060040 .  
## Quartiers_Nord_Ouest -81.7458    25.0523  -3.263 0.001326 ** 
## ---
## Signif. codes:  0 '***' 0.001 '**' 0.01 '*' 0.05 '.' 0.1 ' ' 1
## 
## Residual standard error: 126.8 on 175 degrees of freedom
## Multiple R-squared:  0.8831, Adjusted R-squared:  0.8797 
## F-statistic: 264.4 on 5 and 175 DF,  p-value: < 2.2e-16
\end{verbatim}

\begin{verbatim}
## 
## Call:
## lm(formula = Loyer_TCC ~ Surface + Charges + Pièces + label_eco + 
##     kWh + kgCO2 + Quartiers_Centre + Quartiers_Nord_Est + Quartiers_Nord_Ouest, 
##     data = annonces)
## 
## Residuals:
##     Min      1Q  Median      3Q     Max 
## -375.88  -73.58    1.08   73.79  696.48 
## 
## Coefficients:
##                       Estimate Std. Error t value Pr(>|t|)    
## (Intercept)           395.6343    38.9189  10.166  < 2e-16 ***
## Surface                 9.1852     1.0178   9.025 3.64e-16 ***
## Charges                 0.9140     0.2629   3.477 0.000643 ***
## Pièces                 23.6423    20.6380   1.146 0.253573    
## label_eco             -13.0154    19.8054  -0.657 0.511959    
## kWh                    -0.3097     0.1512  -2.048 0.042103 *  
## kgCO2                  -0.5603     0.5713  -0.981 0.328123    
## Quartiers_Centre       35.6670    27.2153   1.311 0.191766    
## Quartiers_Nord_Est    -15.2679    27.9475  -0.546 0.585568    
## Quartiers_Nord_Ouest -104.4316    29.5652  -3.532 0.000530 ***
## ---
## Signif. codes:  0 '***' 0.001 '**' 0.01 '*' 0.05 '.' 0.1 ' ' 1
## 
## Residual standard error: 126.9 on 171 degrees of freedom
## Multiple R-squared:  0.8856, Adjusted R-squared:  0.8795 
## F-statistic:   147 on 9 and 171 DF,  p-value: < 2.2e-16
\end{verbatim}

\begin{verbatim}
## 
## Call:
## lm(formula = Loyer_TCC ~ Surface + Charges + kWh + Quartiers_Centre + 
##     Quartiers_Nord_Ouest, data = annonces)
## 
## Residuals:
##     Min      1Q  Median      3Q     Max 
## -397.23  -73.93   -4.26   73.10  704.06 
## 
## Coefficients:
##                      Estimate Std. Error t value Pr(>|t|)    
## (Intercept)          377.8952    33.7860  11.185  < 2e-16 ***
## Surface               10.1402     0.4325  23.444  < 2e-16 ***
## Charges                0.8222     0.2571   3.199 0.001640 ** 
## kWh                   -0.3973     0.1053  -3.775 0.000219 ***
## Quartiers_Centre      44.5086    23.5151   1.893 0.060040 .  
## Quartiers_Nord_Ouest -81.7458    25.0523  -3.263 0.001326 ** 
## ---
## Signif. codes:  0 '***' 0.001 '**' 0.01 '*' 0.05 '.' 0.1 ' ' 1
## 
## Residual standard error: 126.8 on 175 degrees of freedom
## Multiple R-squared:  0.8831, Adjusted R-squared:  0.8797 
## F-statistic: 264.4 on 5 and 175 DF,  p-value: < 2.2e-16
\end{verbatim}

\subsubsection{model complexe selon la graphiqiue choisie par les
méthodes Backward
forwardStepwise}\label{model-complexe-selon-la-graphiqiue-choisie-par-les-muxe9thodes-backward-forwardstepwise}

\begin{verbatim}
## 
## Call:
## lm(formula = Loyer_TCC ~ Surface + Charges + kWh + Quartiers_Centre + 
##     Quartiers_Nord_Ouest, data = annonces)
## 
## Residuals:
##     Min      1Q  Median      3Q     Max 
## -397.23  -73.93   -4.26   73.10  704.06 
## 
## Coefficients:
##                      Estimate Std. Error t value Pr(>|t|)    
## (Intercept)          377.8952    33.7860  11.185  < 2e-16 ***
## Surface               10.1402     0.4325  23.444  < 2e-16 ***
## Charges                0.8222     0.2571   3.199 0.001640 ** 
## kWh                   -0.3973     0.1053  -3.775 0.000219 ***
## Quartiers_Centre      44.5086    23.5151   1.893 0.060040 .  
## Quartiers_Nord_Ouest -81.7458    25.0523  -3.263 0.001326 ** 
## ---
## Signif. codes:  0 '***' 0.001 '**' 0.01 '*' 0.05 '.' 0.1 ' ' 1
## 
## Residual standard error: 126.8 on 175 degrees of freedom
## Multiple R-squared:  0.8831, Adjusted R-squared:  0.8797 
## F-statistic: 264.4 on 5 and 175 DF,  p-value: < 2.2e-16
\end{verbatim}

\begin{verbatim}
## 
## Call:
## lm(formula = Loyer_TCC ~ Surface + Charges + Charges^2 + Pièces + 
##     label_eco + label_eco^2 + kWh + kWh^2 + kgCO2 + kgCO2^2 + 
##     Quartiers_Centre + Quartiers_Nord_Est + Quartiers_Nord_Ouest + 
##     Quartiers_Ouest, data = annonces)
## 
## Residuals:
##     Min      1Q  Median      3Q     Max 
## -375.88  -73.58    1.08   73.79  696.48 
## 
## Coefficients: (1 not defined because of singularities)
##                       Estimate Std. Error t value Pr(>|t|)    
## (Intercept)           395.6343    38.9189  10.166  < 2e-16 ***
## Surface                 9.1852     1.0178   9.025 3.64e-16 ***
## Charges                 0.9140     0.2629   3.477 0.000643 ***
## Pièces                 23.6423    20.6380   1.146 0.253573    
## label_eco             -13.0154    19.8054  -0.657 0.511959    
## kWh                    -0.3097     0.1512  -2.048 0.042103 *  
## kgCO2                  -0.5603     0.5713  -0.981 0.328123    
## Quartiers_Centre       35.6670    27.2153   1.311 0.191766    
## Quartiers_Nord_Est    -15.2679    27.9475  -0.546 0.585568    
## Quartiers_Nord_Ouest -104.4316    29.5652  -3.532 0.000530 ***
## Quartiers_Ouest             NA         NA      NA       NA    
## ---
## Signif. codes:  0 '***' 0.001 '**' 0.01 '*' 0.05 '.' 0.1 ' ' 1
## 
## Residual standard error: 126.9 on 171 degrees of freedom
## Multiple R-squared:  0.8856, Adjusted R-squared:  0.8795 
## F-statistic:   147 on 9 and 171 DF,  p-value: < 2.2e-16
\end{verbatim}

\begin{verbatim}
## 
## Call:
## lm(formula = Loyer_TCC ~ Surface + Charges + kWh + Quartiers_Centre + 
##     Quartiers_Nord_Ouest, data = annonces)
## 
## Residuals:
##     Min      1Q  Median      3Q     Max 
## -397.23  -73.93   -4.26   73.10  704.06 
## 
## Coefficients:
##                      Estimate Std. Error t value Pr(>|t|)    
## (Intercept)          377.8952    33.7860  11.185  < 2e-16 ***
## Surface               10.1402     0.4325  23.444  < 2e-16 ***
## Charges                0.8222     0.2571   3.199 0.001640 ** 
## kWh                   -0.3973     0.1053  -3.775 0.000219 ***
## Quartiers_Centre      44.5086    23.5151   1.893 0.060040 .  
## Quartiers_Nord_Ouest -81.7458    25.0523  -3.263 0.001326 ** 
## ---
## Signif. codes:  0 '***' 0.001 '**' 0.01 '*' 0.05 '.' 0.1 ' ' 1
## 
## Residual standard error: 126.8 on 175 degrees of freedom
## Multiple R-squared:  0.8831, Adjusted R-squared:  0.8797 
## F-statistic: 264.4 on 5 and 175 DF,  p-value: < 2.2e-16
\end{verbatim}

\begin{verbatim}
##             R2 Adj_R2    AIC    BIC Resid_SE
## modele1 0.8856 0.8795 2278.7 2313.8  126.895
## modele2 0.8816 0.8753 -168.8 -133.6    0.147
## modele3 0.8915 0.8837 2275.0 2319.8  124.665
## modele4 0.8334 0.8246 2346.7 2381.9  153.125
\end{verbatim}

\subsection{4.2 Robustesse des
résultats}\label{robustesse-des-ruxe9sultats}

Hypothèses des MCO - nomalités / hétéroscédasticité / \ldots{}

\subsection{4.2 Analyses
complémentaires}\label{analyses-compluxe9mentaires}

\begin{itemize}
\item
  Transformation de variables / effets non linéaires
\item
  Changement structurel
\item
  Effets croisés de variables
\end{itemize}

\section{\texorpdfstring{\textbf{5. Conclusion générale et
perspectives}}{5. Conclusion générale et perspectives}}\label{conclusion-guxe9nuxe9rale-et-perspectives}

Rappel de la problématique et de la méthodologie

Synthèse des résultats principaux

Contributions du mémoire, apports scientifiques et/ou applicatifs

Perspectives d'amélioration ou de recherche future

\section{Bibliographie - Source des
données}\label{bibliographie---source-des-donnuxe9es}

\begin{center}\rule{0.5\linewidth}{0.5pt}\end{center}

\begin{center}\rule{0.5\linewidth}{0.5pt}\end{center}

\end{document}
