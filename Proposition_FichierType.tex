% Options for packages loaded elsewhere
\PassOptionsToPackage{unicode}{hyperref}
\PassOptionsToPackage{hyphens}{url}
%
\documentclass[
]{article}
\usepackage{amsmath,amssymb}
\usepackage{iftex}
\ifPDFTeX
  \usepackage[T1]{fontenc}
  \usepackage[utf8]{inputenc}
  \usepackage{textcomp} % provide euro and other symbols
\else % if luatex or xetex
  \usepackage{unicode-math} % this also loads fontspec
  \defaultfontfeatures{Scale=MatchLowercase}
  \defaultfontfeatures[\rmfamily]{Ligatures=TeX,Scale=1}
\fi
\usepackage{lmodern}
\ifPDFTeX\else
  % xetex/luatex font selection
\fi
% Use upquote if available, for straight quotes in verbatim environments
\IfFileExists{upquote.sty}{\usepackage{upquote}}{}
\IfFileExists{microtype.sty}{% use microtype if available
  \usepackage[]{microtype}
  \UseMicrotypeSet[protrusion]{basicmath} % disable protrusion for tt fonts
}{}
\makeatletter
\@ifundefined{KOMAClassName}{% if non-KOMA class
  \IfFileExists{parskip.sty}{%
    \usepackage{parskip}
  }{% else
    \setlength{\parindent}{0pt}
    \setlength{\parskip}{6pt plus 2pt minus 1pt}}
}{% if KOMA class
  \KOMAoptions{parskip=half}}
\makeatother
\usepackage{xcolor}
\usepackage[margin=1in]{geometry}
\usepackage{color}
\usepackage{fancyvrb}
\newcommand{\VerbBar}{|}
\newcommand{\VERB}{\Verb[commandchars=\\\{\}]}
\DefineVerbatimEnvironment{Highlighting}{Verbatim}{commandchars=\\\{\}}
% Add ',fontsize=\small' for more characters per line
\usepackage{framed}
\definecolor{shadecolor}{RGB}{255,255,255}
\newenvironment{Shaded}{\begin{snugshade}}{\end{snugshade}}
\newcommand{\AlertTok}[1]{\textcolor[rgb]{0.75,0.01,0.01}{\textbf{\colorbox[rgb]{0.97,0.90,0.90}{#1}}}}
\newcommand{\AnnotationTok}[1]{\textcolor[rgb]{0.79,0.38,0.79}{#1}}
\newcommand{\AttributeTok}[1]{\textcolor[rgb]{0.00,0.34,0.68}{#1}}
\newcommand{\BaseNTok}[1]{\textcolor[rgb]{0.69,0.50,0.00}{#1}}
\newcommand{\BuiltInTok}[1]{\textcolor[rgb]{0.39,0.29,0.61}{\textbf{#1}}}
\newcommand{\CharTok}[1]{\textcolor[rgb]{0.57,0.30,0.62}{#1}}
\newcommand{\CommentTok}[1]{\textcolor[rgb]{0.54,0.53,0.53}{#1}}
\newcommand{\CommentVarTok}[1]{\textcolor[rgb]{0.00,0.58,1.00}{#1}}
\newcommand{\ConstantTok}[1]{\textcolor[rgb]{0.67,0.33,0.00}{#1}}
\newcommand{\ControlFlowTok}[1]{\textcolor[rgb]{0.12,0.11,0.11}{\textbf{#1}}}
\newcommand{\DataTypeTok}[1]{\textcolor[rgb]{0.00,0.34,0.68}{#1}}
\newcommand{\DecValTok}[1]{\textcolor[rgb]{0.69,0.50,0.00}{#1}}
\newcommand{\DocumentationTok}[1]{\textcolor[rgb]{0.38,0.47,0.50}{#1}}
\newcommand{\ErrorTok}[1]{\textcolor[rgb]{0.75,0.01,0.01}{\underline{#1}}}
\newcommand{\ExtensionTok}[1]{\textcolor[rgb]{0.00,0.58,1.00}{\textbf{#1}}}
\newcommand{\FloatTok}[1]{\textcolor[rgb]{0.69,0.50,0.00}{#1}}
\newcommand{\FunctionTok}[1]{\textcolor[rgb]{0.39,0.29,0.61}{#1}}
\newcommand{\ImportTok}[1]{\textcolor[rgb]{1.00,0.33,0.00}{#1}}
\newcommand{\InformationTok}[1]{\textcolor[rgb]{0.69,0.50,0.00}{#1}}
\newcommand{\KeywordTok}[1]{\textcolor[rgb]{0.12,0.11,0.11}{\textbf{#1}}}
\newcommand{\NormalTok}[1]{\textcolor[rgb]{0.12,0.11,0.11}{#1}}
\newcommand{\OperatorTok}[1]{\textcolor[rgb]{0.12,0.11,0.11}{#1}}
\newcommand{\OtherTok}[1]{\textcolor[rgb]{0.00,0.43,0.16}{#1}}
\newcommand{\PreprocessorTok}[1]{\textcolor[rgb]{0.00,0.43,0.16}{#1}}
\newcommand{\RegionMarkerTok}[1]{\textcolor[rgb]{0.00,0.34,0.68}{\colorbox[rgb]{0.88,0.91,0.97}{#1}}}
\newcommand{\SpecialCharTok}[1]{\textcolor[rgb]{0.24,0.68,0.91}{#1}}
\newcommand{\SpecialStringTok}[1]{\textcolor[rgb]{1.00,0.33,0.00}{#1}}
\newcommand{\StringTok}[1]{\textcolor[rgb]{0.75,0.01,0.01}{#1}}
\newcommand{\VariableTok}[1]{\textcolor[rgb]{0.00,0.34,0.68}{#1}}
\newcommand{\VerbatimStringTok}[1]{\textcolor[rgb]{0.75,0.01,0.01}{#1}}
\newcommand{\WarningTok}[1]{\textcolor[rgb]{0.75,0.01,0.01}{#1}}
\usepackage{longtable,booktabs,array}
\usepackage{calc} % for calculating minipage widths
% Correct order of tables after \paragraph or \subparagraph
\usepackage{etoolbox}
\makeatletter
\patchcmd\longtable{\par}{\if@noskipsec\mbox{}\fi\par}{}{}
\makeatother
% Allow footnotes in longtable head/foot
\IfFileExists{footnotehyper.sty}{\usepackage{footnotehyper}}{\usepackage{footnote}}
\makesavenoteenv{longtable}
\usepackage{graphicx}
\makeatletter
\def\maxwidth{\ifdim\Gin@nat@width>\linewidth\linewidth\else\Gin@nat@width\fi}
\def\maxheight{\ifdim\Gin@nat@height>\textheight\textheight\else\Gin@nat@height\fi}
\makeatother
% Scale images if necessary, so that they will not overflow the page
% margins by default, and it is still possible to overwrite the defaults
% using explicit options in \includegraphics[width, height, ...]{}
\setkeys{Gin}{width=\maxwidth,height=\maxheight,keepaspectratio}
% Set default figure placement to htbp
\makeatletter
\def\fps@figure{htbp}
\makeatother
\setlength{\emergencystretch}{3em} % prevent overfull lines
\providecommand{\tightlist}{%
  \setlength{\itemsep}{0pt}\setlength{\parskip}{0pt}}
\setcounter{secnumdepth}{5}
\ifLuaTeX
  \usepackage{selnolig}  % disable illegal ligatures
\fi
\usepackage{bookmark}
\IfFileExists{xurl.sty}{\usepackage{xurl}}{} % add URL line breaks if available
\urlstyle{same}
\hypersetup{
  pdftitle={Analyse des déterminants du loyer à Rennes},
  pdfauthor={Nom Prénom},
  hidelinks,
  pdfcreator={LaTeX via pandoc}}

\title{Analyse des déterminants du loyer à Rennes}
\author{Nom Prénom}
\date{2025-11-26}

\begin{document}
\maketitle

\textbf{NOM - Prénom}

\href{https://eco.univ-rennes.fr/master-mathematiques-appliquees-statistique}{\textbf{Master
1 MAS - Université de Rennes}}

\textbf{ou}

\href{https://formations.univ-rennes2.fr/fr/formations/master-37/master-mention-mathematiques-appliquees-statistique-parcours-sciences-des-donnees-intelligence-artificielle-JFTJBMKM.html}{\textbf{Master
1 MAS - Université Rennes 2}}

{Pensez à numéroter vos tables et graphiques, cela vous permet de faire
des rappels dans votre texte. Attention également au titre et à la bonne
définition des axes. Chaque graphique est commenté, il est parfois
préférable de les regrouper au sein d'une même fenêtre pour que le
rapport soit plus agréable à lire. N'hésitez pas à faire des sélections
de graphiques intéressants plutôt que de les présenter d'une manière
mécanique. Il ne faut pas utiliser le code de vos variables dans vos
commentaires mais revenir sur leur signification.}

\section{1. Introduction}\label{introduction}

L'accès au logement constitue aujourd'hui un enjeu socio-économique
majeur en France. Dans un contexte de pression croissante sur le marché
locatif, notamment dans les villes à forte attractivité universitaire
comme Rennes, les ménages sont confrontés à une hausse soutenue des
loyers. Comprendre les facteurs qui influencent le prix des logements
devient alors essentiel pour analyser les disparités territoriales et
éclairer les décisions des acteurs publics.

Dans ce projet, nous cherchons à analyser empiriquement les
\textbf{déterminants du loyer tout compris} des logements situés à
Rennes, en utilisant des données collectées automatiquement par web
scraping sur le site \emph{Ouest-France Immo}. L'étude se concentre sur
les logements qui sont répartis dans quatre grands secteurs de Rennes :
\textbf{Centre}, \textbf{Ouest}, \textbf{Nord-Est} et
\textbf{Nord-Ouest}.

\subsection{1.1 Contexte et
problématique}\label{contexte-et-probluxe9matique}

Le marché rennais est particulièrement dynamique en raison de sa
croissance démographique, de son attractivité étudiante et d'une offre
locative limitée. Dès lors, une question centrale se pose :

\begin{quote}
\textbf{Quels facteurs structurels expliquent les variations de loyer à
Rennes ?}
\end{quote}

Cela conduit à examiner l'effet de la surface, du nombre de pièces, des
charges, de la performance énergétique et de la localisation
géographique.

\subsection{1.2 Motivation et intérêt du
sujet}\label{motivation-et-intuxe9ruxeat-du-sujet}

Ce travail présente plusieurs intérêts : - illustrer une application
réelle de la \textbf{régression linéaire} à partir de données collectées
soi-même ; - analyser les \textbf{disparités intra-urbaines} du marché
locatif rennais ; - évaluer l'influence de caractéristiques
structurelles ou environnementales du logement ; - proposer un exemple
complet de démarche allant de la collecte de données au modèle final.

\subsection{1.3 Questions de recherche et
hypothèses}\label{questions-de-recherche-et-hypothuxe8ses}

Les questions principales sont :

\begin{enumerate}
\def\labelenumi{\arabic{enumi}.}
\tightlist
\item
  La surface influence-t-elle fortement le montant du loyer ?
\item
  Les charges intégrées modifient-elles significativement le niveau du
  loyer tout compris ?
\item
  La performance énergétique joue-t-elle un rôle dans la détermination
  du prix ?
\item
  Le quartier exerce-t-il un effet propre sur les niveaux de loyer ?
\end{enumerate}

Les hypothèses qui en découlent sont :

\begin{itemize}
\item
  \textbf{H1 :} une surface plus élevée entraîne un loyer plus élevé
\item
  \textbf{H2 :} des charges plus importantes augmentent le loyer tout
  compris
\item
  \textbf{H3 :} une meilleure performance énergétique est associée à un
  loyer supérieur
\item
  \textbf{H4 :} les logements du Centre sont plus chers que ceux des
  autres quartiers.
\end{itemize}

\subsection{1.4 Objectifs du mémoire}\label{objectifs-du-muxe9moire}

Les objectifs du projet sont : - modéliser et expliquer le loyer mensuel
tout compris ; - identifier les déterminants les plus significatifs ; -
analyser les différences selon les quartiers rennais ; - proposer une
interprétation économétrique rigoureuse.

\subsection{1.5 Plan du document}\label{plan-du-document}

Ce mémoire est structuré comme suit :

\begin{itemize}
\tightlist
\item
  \textbf{Section 1 : Introduction}
\item
  \textbf{Section 2 : Cadre méthodologique}
\item
  \textbf{Section 3 : Analyse descriptive des données}
\item
  \textbf{Section 4 : Estimation et interprétation du modèle}
\item
  \textbf{Section 5 : Conclusion}
\end{itemize}

\begin{center}\rule{0.5\linewidth}{0.5pt}\end{center}

\section{2. Cadre méthodologique}\label{cadre-muxe9thodologique}

Cette section présente le modèle économétrique utilisé et les choix
méthodologiques qui guident l'estimation.

\subsection{2.1 Choix méthodologiques : régression linéaire et
MCO}\label{choix-muxe9thodologiques-ruxe9gression-linuxe9aire-et-mco}

Le modèle retenu est une \textbf{régression linéaire multiple}, estimée
à l'aide de la méthode des \textbf{Moindres Carrés Ordinaires (MCO)}.
Cette approche permet d'étudier comment une variable quantitative ---
ici le loyer tout compris --- dépend de plusieurs caractéristiques
propres au logement.

La méthode MCO consiste à minimiser la somme des carrés des écarts entre
les valeurs observées et celles prédites par le modèle. Cette méthode
est privilégiée pour sa simplicité, son interprétabilité et les
propriétés de ses estimateurs.

\subsection{2.2 Formulation du modèle}\label{formulation-du-moduxe8le}

La variable expliquée est le \textbf{loyer mensuel tout compris} (en
euros).

Les variables explicatives sont :

\begin{itemize}
\tightlist
\item
  la \textbf{surface} du logement (m²),
\item
  les \textbf{charges comprises} (euros),
\item
  le \textbf{nombre de pièces},
\item
  la \textbf{classe énergétique} (A, B, C, \ldots),
\item
  la \textbf{consommation énergétique annuelle} (kWh/an),
\item
  les \textbf{émissions de CO₂} (kg/an),
\item
  le \textbf{quartier} (Centre, Ouest, Nord-Est, Nord-Ouest).
\end{itemize}

Le modèle général s'écrit :

\[
\text{Loyer}_i = \beta_0 +
\beta_1 \text{Surface}_i +
\beta_2 \text{Charges}_i +
\beta_3 \text{Pièces}_i +
\beta_4 \text{ClasseÉnergie}_i +
\beta_5 \text{Consommation}_i +
\beta_6 \text{ÉmissionsCO2}_i +
\sum_q \gamma_q \text{Quartier}_{qi} +
\varepsilon_i
\]

où : - \(\beta_0\) est la constante

\begin{itemize}
\item
  \(\beta_j\) mesure l'effet marginal de chaque caractéristique
\item
  \(\gamma_q\) capture l'effet des quartiers par rapport à un quartier
  de référence
\item
  \(\varepsilon_i\) est le terme d'erreur
\end{itemize}

\subsection{2.3 Justification des choix
techniques}\label{justification-des-choix-techniques}

Plusieurs éléments motivent ce cadre :

\begin{itemize}
\tightlist
\item
  La régression linéaire est un outil adapté pour analyser les facteurs
  influençant une variable économique.
\item
  La méthode MCO fournit des estimateurs robustes et facilement
  interprétables.
\item
  L'utilisation de \textbf{variables indicatrices} pour les quartiers
  permet de mettre en évidence les différences géographiques de prix.
\item
  L'intégration des variables énergétiques répond aux enjeux actuels de
  performance environnementale.
\item
  Le modèle permet de mesurer l'effet propre de chaque variable tout en
  contrôlant les autres.
\end{itemize}

\begin{center}\rule{0.5\linewidth}{0.5pt}\end{center}

\section{3. Analyse des données}\label{analyse-des-donnuxe9es}

\subsection{3.1 Cadre des données}\label{cadre-des-donnuxe9es}

Les données proviennent du site \textbf{Ouest-France Immo}, collectées à
l'aide d'un script de web scraping. Elles décrivent des logements mis en
location dans la \textbf{ville de Rennes}, .

L'étude porte sur \textbf{quatre secteurs géographiques} :

\begin{itemize}
\item
  Quartiers Centre
\item
  Quartiers Ouest
\item
  Quartiers Nord-Est
\item
  Quartiers Nord-Ouest
\end{itemize}

Chaque ligne du jeu de données correspond à \textbf{une annonce
unique}.\\
Les variables collectées décrivent :

\begin{itemize}
\item
  Les caractéristiques du logement (surface, nombre de pièces, charges)
\item
  Sa performance énergétique (classe, consommation, émissions)
\item
  Sa localisation.
\end{itemize}

\subsubsection{Importation des
données}\label{importation-des-donnuxe9es}

\begin{verbatim}
## Warning: le package 'tidyverse' a été compilé avec la version R 4.4.3
\end{verbatim}

\begin{verbatim}
## Warning: le package 'ggplot2' a été compilé avec la version R 4.4.2
\end{verbatim}

\begin{verbatim}
## Warning: le package 'tidyr' a été compilé avec la version R 4.4.3
\end{verbatim}

\begin{verbatim}
## Warning: le package 'readr' a été compilé avec la version R 4.4.2
\end{verbatim}

\begin{verbatim}
## Warning: le package 'purrr' a été compilé avec la version R 4.4.3
\end{verbatim}

\begin{verbatim}
## Warning: le package 'dplyr' a été compilé avec la version R 4.4.3
\end{verbatim}

\begin{verbatim}
## Warning: le package 'forcats' a été compilé avec la version R 4.4.3
\end{verbatim}

\begin{verbatim}
## Warning: le package 'lubridate' a été compilé avec la version R 4.4.3
\end{verbatim}

\begin{verbatim}
## -- Attaching core tidyverse packages ------------------------ tidyverse 2.0.0 --
## v dplyr     1.1.4     v readr     2.1.5
## v forcats   1.0.0     v stringr   1.5.1
## v ggplot2   3.5.1     v tibble    3.2.1
## v lubridate 1.9.4     v tidyr     1.3.1
## v purrr     1.0.4     
## -- Conflicts ------------------------------------------ tidyverse_conflicts() --
## x dplyr::filter() masks stats::filter()
## x dplyr::lag()    masks stats::lag()
## i Use the conflicted package (<http://conflicted.r-lib.org/>) to force all conflicts to become errors
\end{verbatim}

\subsubsection{Tableau 1 -- Description des
variables}\label{tableau-1-description-des-variables}

\begin{longtable}[]{@{}
  >{\raggedright\arraybackslash}p{(\columnwidth - 6\tabcolsep) * \real{0.1700}}
  >{\raggedright\arraybackslash}p{(\columnwidth - 6\tabcolsep) * \real{0.3400}}
  >{\raggedright\arraybackslash}p{(\columnwidth - 6\tabcolsep) * \real{0.1600}}
  >{\raggedright\arraybackslash}p{(\columnwidth - 6\tabcolsep) * \real{0.3300}}@{}}
\caption{Tableau 1 : Description des variables}\tabularnewline
\toprule\noalign{}
\begin{minipage}[b]{\linewidth}\raggedright
Code.de.la.série
\end{minipage} & \begin{minipage}[b]{\linewidth}\raggedright
Définition
\end{minipage} & \begin{minipage}[b]{\linewidth}\raggedright
Unité
\end{minipage} & \begin{minipage}[b]{\linewidth}\raggedright
Source
\end{minipage} \\
\midrule\noalign{}
\endfirsthead
\toprule\noalign{}
\begin{minipage}[b]{\linewidth}\raggedright
Code.de.la.série
\end{minipage} & \begin{minipage}[b]{\linewidth}\raggedright
Définition
\end{minipage} & \begin{minipage}[b]{\linewidth}\raggedright
Unité
\end{minipage} & \begin{minipage}[b]{\linewidth}\raggedright
Source
\end{minipage} \\
\midrule\noalign{}
\endhead
\bottomrule\noalign{}
\endlastfoot
prix\_TCC & Loyer mensuel tout compris & Euros & Ouest-France Immo (web
scraping) \\
Surface & Surface habitable du logement & m² & Ouest-France Immo (web
scraping) \\
Charges & Montant des charges mensuelles & Euros & Ouest-France Immo
(web scraping) \\
Pieces & Nombre de pièces du logement & Nombre & Ouest-France Immo (web
scraping) \\
label\_eco & Classe énergétique du logement & Catégorie (A--G) &
Ouest-France Immo (web scraping) \\
kWh & Consommation énergétique annuelle & kWh/an & Ouest-France Immo
(web scraping) \\
kgCO2 & Émissions annuelles de CO2 & kg/an & Ouest-France Immo (web
scraping) \\
Quartiers & Secteur géographique du logement & Nom du quartier &
Ouest-France Immo (web scraping) \\
\end{longtable}

\subsection{3.2 Prétraitement et nettoyage des
données}\label{pruxe9traitement-et-nettoyage-des-donnuxe9es}

Avant de procéder aux analyses descriptives et économétriques, il a été
nécessaire de réaliser un travail approfondi de préparation des données
issues du web scraping. Comme souvent avec des données collectées
directement sur un site d'annonces, plusieurs problèmes se présentent :
valeurs manquantes, incohérences, doublons, hétérogénéité dans les
formats ou dans les catégories. Cette section présente de manière
structurée l'ensemble des opérations réalisées pour obtenir un jeu de
données exploitable.

\subsubsection{Sélection et renommage des
variables}\label{suxe9lection-et-renommage-des-variables}

Dans un premier temps, seules les variables pertinentes pour l'étude ont
été conservées.\\
Les colonnes correspondant ``Loyer'', ``Surface.habitable'',
``Dont.charges'',``Pièces'' , ``Pièce'', ``label\_eco'',
``kWh.m\ldots an'',``kgCO2.m\ldots an'', ``Quartiers'' ont été
extraites.\\
Les noms ont également été standardisés pour faciliter leur manipulation
(par exemple : renommer \emph{Surface habitable} en \emph{Surface},
\emph{Dont charges} en \emph{Charges}, etc.).

\subsubsection{Découper le Loyer TTC en
facteur}\label{duxe9couper-le-loyer-ttc-en-facteur}

\begin{Shaded}
\begin{Highlighting}[]
\NormalTok{annonces}\SpecialCharTok{$}\NormalTok{Loyer\_Groupe }\OtherTok{\textless{}{-}} \FunctionTok{cut}\NormalTok{(annonces}\SpecialCharTok{$}\NormalTok{Loyer\_TTC , }\AttributeTok{breaks =} \FunctionTok{quantile}\NormalTok{(annonces}\SpecialCharTok{$}\NormalTok{Loyer\_TTC ,}\AttributeTok{probs =} \FunctionTok{seq}\NormalTok{(}\DecValTok{0}\NormalTok{, }\DecValTok{1}\NormalTok{, }\FloatTok{0.25}\NormalTok{)) , }\AttributeTok{include.lowest =} \ConstantTok{TRUE}\NormalTok{)}
\end{Highlighting}
\end{Shaded}

\subsubsection{Gestion des valeurs
manquantes}\label{gestion-des-valeurs-manquantes}

La base contenait un certain nombre d'observations incomplètes.
Plusieurs traitements ont été appliqués :

\begin{itemize}
\item
  \textbf{Correction des pièces manquantes} : Le jeu de données contient
  deux variables liées au nombre de pièces du logement :
\item
  \texttt{Pièces} : prend une valeur manquante (\texttt{NA}) ou bien un
  nombre de pièces \textbf{strictement supérieur à 1} ;
\item
  \texttt{Piece} : variable binaire prenant la valeur \texttt{1} ou
  \texttt{NA}, et renseignée uniquement lorsque \texttt{Pièces} est
  manquante.. Donc , nous avons remplacé tout les valeurs manquantes par
  1 (1 pièce).
\end{itemize}

\begin{Shaded}
\begin{Highlighting}[]
\NormalTok{annonces }\OtherTok{\textless{}{-}}\NormalTok{ annonces }\SpecialCharTok{\%\textgreater{}\%} \FunctionTok{mutate}\NormalTok{(Pièces }\OtherTok{=} \FunctionTok{ifelse}\NormalTok{(}\FunctionTok{is.na}\NormalTok{(Pièces) , }\DecValTok{1}\NormalTok{ ,Pièces))}
\NormalTok{annonces}\OtherTok{\textless{}{-}}\NormalTok{ annonces}\SpecialCharTok{\%\textgreater{}\%}\FunctionTok{select}\NormalTok{(}\SpecialCharTok{{-}}\NormalTok{Pièce)}
\end{Highlighting}
\end{Shaded}

\begin{itemize}
\tightlist
\item
  \textbf{Exclusion des annonces non classées énergétiquement} : les
  logements portant la mention \emph{NC} (Non Classé) ont été retirés.
\end{itemize}

\begin{Shaded}
\begin{Highlighting}[]
\NormalTok{annonces}\SpecialCharTok{$}\NormalTok{label\_eco }\OtherTok{\textless{}{-}}\FunctionTok{as.factor}\NormalTok{(annonces}\SpecialCharTok{$}\NormalTok{label\_eco)}
\NormalTok{annonces }\OtherTok{\textless{}{-}}\NormalTok{annonces }\SpecialCharTok{\%\textgreater{}\%}\FunctionTok{filter}\NormalTok{(label\_eco }\SpecialCharTok{!=}\StringTok{"NC"}\NormalTok{)}
\FunctionTok{table}\NormalTok{(annonces}\SpecialCharTok{$}\NormalTok{label\_eco)}
\end{Highlighting}
\end{Shaded}

\begin{verbatim}
## 
##  A  B  C  D  E  F  G NC 
##  2  6 41 73 44 14  1  0
\end{verbatim}

\begin{itemize}
\tightlist
\item
  \textbf{Regroupement de certaines catégories énergétiques} : les
  labels \emph{G}, \emph{A} , \emph{B} et \emph{C} ont été regroupés
  dans une catégorie commune (\emph{C+}) et les labels \emph{E} et
  \emph{F} ont été regroupés dans une catégorie commune (\emph{E-}) pour
  éviter des classes trop peu représentées.
\end{itemize}

\begin{Shaded}
\begin{Highlighting}[]
\FunctionTok{levels}\NormalTok{(annonces}\SpecialCharTok{$}\NormalTok{label\_eco)[}\FunctionTok{levels}\NormalTok{(annonces}\SpecialCharTok{$}\NormalTok{label\_eco) }\SpecialCharTok{\%in\%} \FunctionTok{c}\NormalTok{(}\StringTok{"G"}\NormalTok{,}\StringTok{"A"}\NormalTok{,}\StringTok{"B"}\NormalTok{ ,}\StringTok{"C"}\NormalTok{)] }\OtherTok{\textless{}{-}}\StringTok{"C+"}
\FunctionTok{levels}\NormalTok{(annonces}\SpecialCharTok{$}\NormalTok{label\_eco)[}\FunctionTok{levels}\NormalTok{(annonces}\SpecialCharTok{$}\NormalTok{label\_eco) }\SpecialCharTok{\%in\%} \FunctionTok{c}\NormalTok{(}\StringTok{"E"}\NormalTok{,}\StringTok{"F"}\NormalTok{)] }\OtherTok{\textless{}{-}}\StringTok{"E{-}"}

\FunctionTok{table}\NormalTok{(annonces}\SpecialCharTok{$}\NormalTok{label\_eco)}
\end{Highlighting}
\end{Shaded}

\begin{verbatim}
## 
## C+  D E- NC 
## 50 73 58  0
\end{verbatim}

\begin{itemize}
\tightlist
\item
  \textbf{Imputation énergétique} : les valeurs manquantes concernant la
  consommation en kWh et les émissions de CO₂ ont été remplacées par la
  moyenne calculée au sein de chaque classe énergétique.\\
  Cela permet une imputation cohérente avec le profil énergétique du
  logement.
\end{itemize}

Ces opérations permettent d'éviter la perte excessive d'observations
tout en conservant une cohérence structurelle entre les variables.

\begin{Shaded}
\begin{Highlighting}[]
\NormalTok{annonces }\OtherTok{\textless{}{-}}\NormalTok{ annonces }\SpecialCharTok{\%\textgreater{}\%}
  \FunctionTok{group\_by}\NormalTok{(label\_eco) }\SpecialCharTok{\%\textgreater{}\%}
  \FunctionTok{mutate}\NormalTok{(}
    \AttributeTok{kWh   =} \FunctionTok{ifelse}\NormalTok{(}\FunctionTok{is.na}\NormalTok{(kWh),   }\FunctionTok{mean}\NormalTok{(kWh,   }\AttributeTok{na.rm =} \ConstantTok{TRUE}\NormalTok{), kWh),}
    \AttributeTok{kgCO2 =} \FunctionTok{ifelse}\NormalTok{(}\FunctionTok{is.na}\NormalTok{(kgCO2), }\FunctionTok{mean}\NormalTok{(kgCO2, }\AttributeTok{na.rm =} \ConstantTok{TRUE}\NormalTok{), kgCO2)}
\NormalTok{  ) }\SpecialCharTok{\%\textgreater{}\%}
  \FunctionTok{ungroup}\NormalTok{()}
\end{Highlighting}
\end{Shaded}

\subsubsection{Vérification des effectifs et cohérence
interne}\label{vuxe9rification-des-effectifs-et-cohuxe9rence-interne}

Les effectifs par label énergétique ont été inspectés pour vérifier la
répartition des logements après nettoyage.\\
Cette étape permet de s'assurer que les regroupements effectués ne
créent pas de déséquilibres majeurs ou de classes trop petites.

Le travail consiste également à vérifier la cohérence logique des
données, comme :

\begin{itemize}
\tightlist
\item
  absence de valeurs énergétiques impossibles,
\item
  absence de surfaces irréalistes,
\item
  cohérence entre loyer et charges.
\end{itemize}

Les observations incohérentes ont été retirées ou corrigées selon les
cas.

\subsubsection{Remplacer des valeurs manquantes dans les
charges}\label{remplacer-des-valeurs-manquantes-dans-les-charges}

Certaines annonces ne précisaient pas le montant des charges.\\
Plutôt qu'une suppression pure et simple, les charges manquantes ont été
estimées en utilisant une méthode par \textbf{groupes de loyers} :

\begin{enumerate}
\def\labelenumi{\arabic{enumi}.}
\tightlist
\item
  Le loyer total a été divisé en cinq intervalles (du plus faible au
  plus élevé).
\item
  Chaque logement a été associé à un groupe de loyer.
\item
  Les charges manquantes ont été remplacées par la moyenne observée dans
  leur groupe de loyers respectif.
\end{enumerate}

Cette méthode permet d'imputer les charges selon un critère économique
pertinent : les logements ayant

\begin{Shaded}
\begin{Highlighting}[]
\CommentTok{\# Remplacer les charges manquantes par la moyenne du groupe de loyer (arrondi à 2 décimales)}
\NormalTok{annonces }\OtherTok{\textless{}{-}}\NormalTok{ annonces }\SpecialCharTok{\%\textgreater{}\%}
  \FunctionTok{group\_by}\NormalTok{(Loyer\_Groupe) }\SpecialCharTok{\%\textgreater{}\%}
  \FunctionTok{mutate}\NormalTok{(}
    \AttributeTok{Charges =} \FunctionTok{ifelse}\NormalTok{(}
      \FunctionTok{is.na}\NormalTok{(Charges),}
      \FunctionTok{round}\NormalTok{(}\FunctionTok{mean}\NormalTok{(Charges, }\AttributeTok{na.rm =} \ConstantTok{TRUE}\NormalTok{), }\DecValTok{2}\NormalTok{),}
\NormalTok{      Charges}
\NormalTok{    )}
\NormalTok{  ) }\SpecialCharTok{\%\textgreater{}\%}
  \FunctionTok{ungroup}\NormalTok{()}
\end{Highlighting}
\end{Shaded}

\subsubsection{Constitution du jeu de données
final}\label{constitution-du-jeu-de-donnuxe9es-final}

\paragraph{Vérification l'absence de valeur
manquante}\label{vuxe9rification-labsence-de-valeur-manquante}

\begin{Shaded}
\begin{Highlighting}[]
\FunctionTok{anyNA}\NormalTok{(annonces)}
\end{Highlighting}
\end{Shaded}

\begin{verbatim}
## [1] FALSE
\end{verbatim}

Après l'ensemble de ces traitements, la base obtenue présente les
caractéristiques suivantes :

\begin{itemize}
\tightlist
\item
  absence de doublons,
\item
  absence de valeurs cruciales manquantes,
\item
  cohérence des formats (numérique, catégoriel, etc.),
\item
  regroupement homogène des catégories énergétiques,
\item
  estimation raisonnable des charges manquantes,
\item
  structure compatible avec les outils de modélisation et d'analyse
  statistique.
\end{itemize}

Cette base nettoyée constitue le fondement de l'analyse descriptive
(Section 3.3) et du modèle économétrique présenté dans la Section 4.
\#\# 3.3 Analyse descriptive univariée des données Cette section
présente une description statistique des variables utilisées dans le
modèle, après les étapes de nettoyage décrites précédemment.\\
L'objectif est de caractériser la distribution de chacune des variables,
d'identifier les tendances générales et de repérer d'éventuelles valeurs
atypiques.\\
L'analyse s'appuie sur des tableaux de statistiques descriptives ainsi
que des représentations graphiques adaptées (histogrammes, boxplots,
diagrammes en barres). \#\#\# Résumé statistique

Les principales variables quantitatives du jeu de données sont :

\begin{itemize}
\tightlist
\item
  le \textbf{loyer total} (euros),
\item
  la \textbf{surface habitable} (m²),
\item
  les \textbf{charges mensuelles} (euros),
\item
  le \textbf{nombre de pièces},
\item
  la \textbf{consommation énergétique} (kWh/an),
\item
  les \textbf{émissions de CO₂} (kg/an).
\end{itemize}

Le tableau suivant présente les statistiques descriptives classiques :
moyenne, médiane, minimum, maximum et écart-type.

\begin{longtable}[]{@{}
  >{\raggedright\arraybackslash}p{(\columnwidth - 16\tabcolsep) * \real{0.1515}}
  >{\raggedright\arraybackslash}p{(\columnwidth - 16\tabcolsep) * \real{0.1515}}
  >{\raggedleft\arraybackslash}p{(\columnwidth - 16\tabcolsep) * \real{0.0606}}
  >{\raggedleft\arraybackslash}p{(\columnwidth - 16\tabcolsep) * \real{0.0606}}
  >{\raggedleft\arraybackslash}p{(\columnwidth - 16\tabcolsep) * \real{0.1061}}
  >{\raggedleft\arraybackslash}p{(\columnwidth - 16\tabcolsep) * \real{0.1515}}
  >{\raggedleft\arraybackslash}p{(\columnwidth - 16\tabcolsep) * \real{0.0758}}
  >{\raggedleft\arraybackslash}p{(\columnwidth - 16\tabcolsep) * \real{0.0758}}
  >{\raggedleft\arraybackslash}p{(\columnwidth - 16\tabcolsep) * \real{0.1667}}@{}}
\caption{Tableau 1 : Statistiques descriptives des variables
quantitatives}\tabularnewline
\toprule\noalign{}
\begin{minipage}[b]{\linewidth}\raggedright
\end{minipage} & \begin{minipage}[b]{\linewidth}\raggedright
Var
\end{minipage} & \begin{minipage}[b]{\linewidth}\raggedleft
Min
\end{minipage} & \begin{minipage}[b]{\linewidth}\raggedleft
Q1
\end{minipage} & \begin{minipage}[b]{\linewidth}\raggedleft
Median
\end{minipage} & \begin{minipage}[b]{\linewidth}\raggedleft
Mean
\end{minipage} & \begin{minipage}[b]{\linewidth}\raggedleft
Q3
\end{minipage} & \begin{minipage}[b]{\linewidth}\raggedleft
Max
\end{minipage} & \begin{minipage}[b]{\linewidth}\raggedleft
SD
\end{minipage} \\
\midrule\noalign{}
\endfirsthead
\toprule\noalign{}
\begin{minipage}[b]{\linewidth}\raggedright
\end{minipage} & \begin{minipage}[b]{\linewidth}\raggedright
Var
\end{minipage} & \begin{minipage}[b]{\linewidth}\raggedleft
Min
\end{minipage} & \begin{minipage}[b]{\linewidth}\raggedleft
Q1
\end{minipage} & \begin{minipage}[b]{\linewidth}\raggedleft
Median
\end{minipage} & \begin{minipage}[b]{\linewidth}\raggedleft
Mean
\end{minipage} & \begin{minipage}[b]{\linewidth}\raggedleft
Q3
\end{minipage} & \begin{minipage}[b]{\linewidth}\raggedleft
Max
\end{minipage} & \begin{minipage}[b]{\linewidth}\raggedleft
SD
\end{minipage} \\
\midrule\noalign{}
\endhead
\bottomrule\noalign{}
\endlastfoot
Loyer\_TTC & Loyer\_TTC & 278 & 550 & 710.00 & 818.34807 & 1030 & 2250 &
365.603406 \\
Surface & Surface & 9 & 22 & 35.00 & 46.53039 & 68 & 135 & 30.054126 \\
Charges & Charges & 10 & 45 & 68.43 & 75.61122 & 90 & 300 & 49.908030 \\
Pièces & Pièces & 1 & 1 & 2.00 & 2.21547 & 3 & 5 & 1.351124 \\
kWh & kWh & 37 & 159 & 219.00 & 221.73588 & 282 & 739 & 97.036470 \\
kgCO2 & kgCO2 & 1 & 9 & 19.00 & 25.05173 & 37 & 96 & 18.727677 \\
\end{longtable}

\subsubsection{Statistiques descriptives des variables
catégorielles}\label{statistiques-descriptives-des-variables-catuxe9gorielles}

Les variables qualitatives retenues dans l'analyse sont :

le quartier du logement,

la classe énergétique.

Le tableau suivant présente la distribution des effectifs pour ces
variables.

\begin{Shaded}
\begin{Highlighting}[]
\NormalTok{freq\_quartiers }\OtherTok{\textless{}{-}} \FunctionTok{table}\NormalTok{(annonces}\SpecialCharTok{$}\NormalTok{Quartiers)}
\NormalTok{freq\_energie }\OtherTok{\textless{}{-}} \FunctionTok{table}\NormalTok{(annonces}\SpecialCharTok{$}\NormalTok{label\_eco)}
\NormalTok{freq\_Pièces }\OtherTok{\textless{}{-}} \FunctionTok{table}\NormalTok{(annonces}\SpecialCharTok{$}\NormalTok{Pièces)}
\FunctionTok{table}\NormalTok{(annonces}\SpecialCharTok{$}\NormalTok{Loyer\_Groupe)}
\end{Highlighting}
\end{Shaded}

\begin{verbatim}
## 
##        [278,535]        (535,700]      (700,1e+03] (1e+03,2.25e+03] 
##               43               47               44               47
\end{verbatim}

\begin{Shaded}
\begin{Highlighting}[]
\NormalTok{knitr}\SpecialCharTok{::}\FunctionTok{kable}\NormalTok{(}
\FunctionTok{data.frame}\NormalTok{(Nb\_pièces }\OtherTok{=} \FunctionTok{names}\NormalTok{(freq\_Pièces),}
\AttributeTok{Effectif =} \FunctionTok{as.vector}\NormalTok{(freq\_Pièces)),}
\AttributeTok{caption =} \StringTok{"Tableau 1 : Répartition des logements par le nombre de pièce du logement"}
\NormalTok{)}
\end{Highlighting}
\end{Shaded}

\begin{longtable}[]{@{}lr@{}}
\caption{Tableau 1 : Répartition des logements par le nombre de pièce du
logement}\tabularnewline
\toprule\noalign{}
Nb\_pièces & Effectif \\
\midrule\noalign{}
\endfirsthead
\toprule\noalign{}
Nb\_pièces & Effectif \\
\midrule\noalign{}
\endhead
\bottomrule\noalign{}
\endlastfoot
1 & 80 \\
2 & 37 \\
3 & 23 \\
4 & 27 \\
5 & 14 \\
\end{longtable}

\begin{Shaded}
\begin{Highlighting}[]
\NormalTok{knitr}\SpecialCharTok{::}\FunctionTok{kable}\NormalTok{(}
\FunctionTok{data.frame}\NormalTok{(}\AttributeTok{Quartier =} \FunctionTok{names}\NormalTok{(freq\_quartiers),}
\AttributeTok{Effectif =} \FunctionTok{as.vector}\NormalTok{(freq\_quartiers)),}
\AttributeTok{caption =} \StringTok{"Tableau 2 : Répartition des logements par quartier"}
\NormalTok{)}
\end{Highlighting}
\end{Shaded}

\begin{longtable}[]{@{}lr@{}}
\caption{Tableau 2 : Répartition des logements par
quartier}\tabularnewline
\toprule\noalign{}
Quartier & Effectif \\
\midrule\noalign{}
\endfirsthead
\toprule\noalign{}
Quartier & Effectif \\
\midrule\noalign{}
\endhead
\bottomrule\noalign{}
\endlastfoot
Quartiers Centre & 55 \\
Quartiers Nord-Est & 41 \\
Quartiers Nord-Ouest & 42 \\
Quartiers Ouest & 43 \\
\end{longtable}

\begin{Shaded}
\begin{Highlighting}[]
\NormalTok{knitr}\SpecialCharTok{::}\FunctionTok{kable}\NormalTok{(}
\FunctionTok{data.frame}\NormalTok{(Classe\_énergétique }\OtherTok{=} \FunctionTok{names}\NormalTok{(freq\_energie),}
\AttributeTok{Effectif =} \FunctionTok{as.vector}\NormalTok{(freq\_energie)),}
\AttributeTok{caption =} \StringTok{"Tableau 3 : Répartition des logements par classe énergétique"}
\NormalTok{)}
\end{Highlighting}
\end{Shaded}

\begin{longtable}[]{@{}lr@{}}
\caption{Tableau 3 : Répartition des logements par classe
énergétique}\tabularnewline
\toprule\noalign{}
Classe\_énergétique & Effectif \\
\midrule\noalign{}
\endfirsthead
\toprule\noalign{}
Classe\_énergétique & Effectif \\
\midrule\noalign{}
\endhead
\bottomrule\noalign{}
\endlastfoot
C+ & 50 \\
D & 73 \\
E- & 58 \\
NC & 0 \\
\end{longtable}

Commentaire : Ces tableaux permettent de vérifier l'équilibre des
données entre quartiers et catégories énergétiques. Une répartition trop
inégale pourrait influencer les résultats du modèle. \#\#\# Histogrammes
des principales variables

L'histogramme permet de visualiser la distribution des variables
quantitatives. Les graphiques suivants représentent :

\paragraph{La distribution du
Loyer\_TTC,}\label{la-distribution-du-loyer_ttc}

\begin{Shaded}
\begin{Highlighting}[]
\FunctionTok{ggplot}\NormalTok{(annonces, }\FunctionTok{aes}\NormalTok{(}\AttributeTok{x =}\NormalTok{ Loyer\_TTC)) }\SpecialCharTok{+}
\FunctionTok{geom\_histogram}\NormalTok{(}\AttributeTok{bins =} \DecValTok{25}\NormalTok{, }\AttributeTok{fill =} \StringTok{"lightblue"}\NormalTok{, }\AttributeTok{color =} \StringTok{"black"}\NormalTok{) }\SpecialCharTok{+}
\FunctionTok{labs}\NormalTok{(}\AttributeTok{title =} \StringTok{"Graphique 1 : Distribution du loyer tout compris"}\NormalTok{,}
\AttributeTok{x =} \StringTok{"Loyer TTC (euros)"}\NormalTok{, }\AttributeTok{y =} \StringTok{"Effectif"}\NormalTok{)}
\end{Highlighting}
\end{Shaded}

\includegraphics{Proposition_FichierType_files/figure-latex/unnamed-chunk-12-1.pdf}

\paragraph{La distribution des surfaces selon le groupe de
loyer}\label{la-distribution-des-surfaces-selon-le-groupe-de-loyer}

\begin{Shaded}
\begin{Highlighting}[]
\FunctionTok{ggplot}\NormalTok{(annonces, }\FunctionTok{aes}\NormalTok{(}\AttributeTok{x =}\NormalTok{ Loyer\_Groupe, }\AttributeTok{y =}\NormalTok{ Surface)) }\SpecialCharTok{+}
  \FunctionTok{geom\_boxplot}\NormalTok{(}\AttributeTok{fill =} \FunctionTok{c}\NormalTok{(}\StringTok{"lightblue"}\NormalTok{,}\StringTok{"red"}\NormalTok{ , }\StringTok{"yellow"}\NormalTok{ , }\StringTok{"orange"}\NormalTok{), }\AttributeTok{color =} \StringTok{"black"}\NormalTok{) }\SpecialCharTok{+}
  \FunctionTok{labs}\NormalTok{(}
    \AttributeTok{title =} \StringTok{"Graphique : Distribution des surfaces selon le groupe de loyer"}\NormalTok{,}
    \AttributeTok{x =} \StringTok{"Groupes de loyer (euros)"}\NormalTok{,}
    \AttributeTok{y =} \StringTok{"Surface (m²)"}
\NormalTok{  ) }\SpecialCharTok{+}
  \FunctionTok{theme\_minimal}\NormalTok{() }\SpecialCharTok{+}
  \FunctionTok{theme}\NormalTok{(}\AttributeTok{axis.text.x =} \FunctionTok{element\_text}\NormalTok{(}\AttributeTok{angle =} \DecValTok{25}\NormalTok{, }\AttributeTok{hjust =} \DecValTok{1}\NormalTok{))}
\end{Highlighting}
\end{Shaded}

\includegraphics{Proposition_FichierType_files/figure-latex/unnamed-chunk-13-1.pdf}
\textbf{Commentaire :}\\
On observe que la surface tend globalement à augmenter avec le groupe de
loyer.\\
Les logements appartenant au groupe de loyers le plus faible présentent
les surfaces les plus réduites, souvent inférieures à 25 m².\\
À l'inverse, les groupes de loyers les plus élevés se caractérisent par
des surfaces plus importantes, bien que la variabilité demeure élevée.

La présence de valeurs extrêmes dans les groupes intermédiaires suggère
que le loyer n'est pas seulement déterminé par la surface mais également
par d'autres facteurs tels que la localisation, la qualité du logement
ou la performance énergétique.

Cette analyse confirme toutefois une tendance générale : \textbf{les
logements plus grands sont associés à des niveaux de loyer plus élevés},
ce qui est cohérent avec les mécanismes du marché locatif.

\paragraph{La distribution des
charges}\label{la-distribution-des-charges}

\begin{Shaded}
\begin{Highlighting}[]
\FunctionTok{ggplot}\NormalTok{(annonces, }\FunctionTok{aes}\NormalTok{(}\AttributeTok{x =}\NormalTok{ Charges)) }\SpecialCharTok{+}
\FunctionTok{geom\_histogram}\NormalTok{(}\AttributeTok{bins =} \DecValTok{25}\NormalTok{, }\AttributeTok{fill =} \StringTok{"lightpink"}\NormalTok{, }\AttributeTok{color =} \StringTok{"black"}\NormalTok{) }\SpecialCharTok{+}
\FunctionTok{labs}\NormalTok{(}\AttributeTok{title =} \StringTok{"Graphique 3 : Distribution des charges"}\NormalTok{,}
\AttributeTok{x =} \StringTok{"Charges (euros)"}\NormalTok{, }\AttributeTok{y =} \StringTok{"Effectif"}\NormalTok{)}
\end{Highlighting}
\end{Shaded}

\includegraphics{Proposition_FichierType_files/figure-latex/unnamed-chunk-14-1.pdf}

\paragraph{La distribution de Loyer selon nombre de
Pieces.}\label{la-distribution-de-loyer-selon-nombre-de-pieces.}

\begin{Shaded}
\begin{Highlighting}[]
\FunctionTok{ggplot}\NormalTok{(annonces, }\FunctionTok{aes}\NormalTok{(}\AttributeTok{x =} \FunctionTok{factor}\NormalTok{(Pièces), }\AttributeTok{y =}\NormalTok{ Loyer\_TTC)) }\SpecialCharTok{+}
  \FunctionTok{geom\_boxplot}\NormalTok{(}\AttributeTok{fill =} \StringTok{"lightgreen"}\NormalTok{, }\AttributeTok{color =} \StringTok{"black"}\NormalTok{) }\SpecialCharTok{+}
  \FunctionTok{labs}\NormalTok{(}
    \AttributeTok{title =} \StringTok{"Graphique : Distribution du loyer selon le nombre de pièces"}\NormalTok{,}
    \AttributeTok{x =} \StringTok{"Nombre de pièces"}\NormalTok{,}
    \AttributeTok{y =} \StringTok{"Loyer tout compris (euros)"}
\NormalTok{  ) }\SpecialCharTok{+}
  \FunctionTok{theme\_minimal}\NormalTok{()}
\end{Highlighting}
\end{Shaded}

\includegraphics{Proposition_FichierType_files/figure-latex/unnamed-chunk-15-1.pdf}
\#\#\#\# Boxplots des variables continues

Les boxplots permettent d'identifier visuellement les valeurs extrêmes
(outliers) ainsi que la dispersion de chaque variable

\begin{Shaded}
\begin{Highlighting}[]
\FunctionTok{ggplot}\NormalTok{(annonces, }\FunctionTok{aes}\NormalTok{(}\AttributeTok{y =}\NormalTok{ Loyer\_TTC)) }\SpecialCharTok{+}
\FunctionTok{geom\_boxplot}\NormalTok{(}\AttributeTok{fill =} \StringTok{"orange"}\NormalTok{) }\SpecialCharTok{+}
\FunctionTok{labs}\NormalTok{(}\AttributeTok{title =} \StringTok{"Graphique 4 : Boxplot du loyer total"}\NormalTok{,}
\AttributeTok{y =} \StringTok{"Loyer TTC (euros)"}\NormalTok{)}
\end{Highlighting}
\end{Shaded}

\includegraphics{Proposition_FichierType_files/figure-latex/unnamed-chunk-16-1.pdf}

\begin{Shaded}
\begin{Highlighting}[]
\FunctionTok{ggplot}\NormalTok{(annonces, }\FunctionTok{aes}\NormalTok{(}\AttributeTok{y =}\NormalTok{ Surface)) }\SpecialCharTok{+}
\FunctionTok{geom\_boxplot}\NormalTok{(}\AttributeTok{fill =} \StringTok{"skyblue"}\NormalTok{) }\SpecialCharTok{+}
\FunctionTok{labs}\NormalTok{(}\AttributeTok{title =} \StringTok{"Graphique 5 : Boxplot des surfaces"}\NormalTok{,}
\AttributeTok{y =} \StringTok{"Surface (m²)"}\NormalTok{)}
\end{Highlighting}
\end{Shaded}

\includegraphics{Proposition_FichierType_files/figure-latex/unnamed-chunk-16-2.pdf}

\begin{center}\rule{0.5\linewidth}{0.5pt}\end{center}

\end{document}
